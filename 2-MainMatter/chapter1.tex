%!TEX root = ../template.tex
%%%%%%%%%%%%%%%%%%%%%%%%%%%%%%%%%%%%%%%%%%%%%%%%%%%%%%%%%%%%%%%%%%%
%% chapter1.tex
%% NOVA thesis document file
%%
%% Chapter with introduction
%%%%%%%%%%%%%%%%%%%%%%%%%%%%%%%%%%%%%%%%%%%%%%%%%%%%%%%%%%%%%%%%%%%

\typeout{NT FILE chapter1.tex}%

\chapter{Introduction}
\label{cha:introduction}

\prependtographicspath{{5-Figures/Covers/}}

\glsresetall

The proliferation of \gls{iot} devices has changed how we
monitor and interact with physical spaces -- from smart homes to industrial
facilities -- with the global \gls{iot} market reaching 18.5 billion connected
devices in 2024, and projected to reach 39 billion by
2030~\cite{sinha2025stateofiot}, with industry analysts estimating that \gls{iot}
technologies could unlock between \$5.5 and \$12.6 trillion in economic value
by the same year~\cite{mckinsey2021iot}.

However, current \gls{iot} systems remain fundamentally dependent on continuous
Internet connectivity and centralized cloud
infrastructures~\cite{8123913,s20226441}. The dominant architectural pattern
across the industry relies on edge devices collecting data and transmitting it
to remote cloud servers for processing, storage, and execution of other control
logic. This approach delivers scalability and ease of management but creates a
fundamental dependency on network availability.

This centralized model introduces several concerns beyond just connectivity:
operational costs for cloud services create economic dependencies on
third-party providers and often lead to vendor lock-in, data sovereignty issues
arise when sensitive information must travel to and reside on external
infrastructure, and system resilience becomes fundamentally tied to the
availability of these remote services.

While cloud platforms can benefit \gls{iot} deployments by providing additional
computational resources and storage capacity, systems that \emph{depend} on
constant cloud connectivity sacrifice local autonomy and introduce single
points of failure. This cloud-centric model has enabled rapid \gls{iot} adoption, but
introduces critical vulnerabilities across scenarios where continuous
connectivity cannot be guaranteed.

Recent infrastructure failures illustrate the fragility of cloud-dependent
architectures: the AWS US-EAST-1 outage in October
2025~\cite{thousandeyes2025aws} disrupted services from banking to smart homes
worldwide, demonstrating how a \gls{DNS} configuration error in a \emph{single
	region} can have cascading effects, leading to global
disruptions~\cite{aws2025summary}.

Attempts have been made to address these challenges through incremental
improvements, such as shifting focus to edge computing in order to reduce
latency~\cite{s20226441,8123913,8664595}, redundant cloud regions for
availability~\cite{mesbahi2018reliability,maciel2022survey}, and hybrid
architectures that combine local and remote
processing~\cite{9846938,s24165320,KREKOVIC2025101553}.

However, these solutions remain fundamentally tied to the assumption of
eventual connectivity, and often increase system complexity without eliminating
the core dependency. This leads to shortcomings in key areas that demand
more fundamental architectural reconsideration:

\begin{description}
    \item[\textbf{Limited suitability for hazardous environments:}]
        Remote or dangerous locations (industrial sites, disaster-prone areas)
        require systems that can operate reliably without constant human
        intervention or stable network infrastructure;
    \item[\textbf{Lack of autonomous operation:}]
        Device deployment and operation en masse can be brittle, with little
        tolerance for individual node failures in the \gls{iot} infrastructure,
        which is naturally susceptible to network failures and intermittent
        connectivity;
    \item[\textbf{Privacy and data sovereignty concerns:}]
        Cloud platforms and other third-parties are oftentimes an unavoidable
        middle layer between end devices and end users, raising questions about
        data processing and control.
\end{description}

% \begin{description}
%     \item \textbf{Limited suitability for hazardous environments:}\\
%         Remote or dangerous locations (industrial sites, disaster-prone areas)
%         require systems that can operate reliably without constant human
%         intervention or stable network infrastructure;
%     \item \textbf{Lack of autonomous operation:}\\
%         Device deployment and operation en masse can be brittle, with little
%         tolerance for individual node failures in the \gls{iot} infrastructure,
%         which is naturally susceptible to network failures and intermittent
%         connectivity;
%     \item \textbf{Privacy and data sovereignty concerns:}\\
%         Cloud platforms and other third-parties are oftentimes an unavoidable
%         middle layer between end devices and end users, raising questions about
%         data processing and control.
% \end{description}

While these challenges affect both \gls{iot} and domotics systems alike --
from industrial monitoring to residential automation -- they become
particularly acute in disaster response and emergency scenarios, where
communication infrastructure might become completely unavailable precisely
when needed most, rapid deployment with minimal configuration becomes
essential, and autonomous operation transitions from desirable to
indispensable.

During earthquakes, floods, and natural or human-driven disasters, the need for
real-time sensor data (structural integrity, air quality, evacuation routes)
and bidirectional communication (threat alerts, user feedback) becomes
critical, yet traditional infrastructure often fails first, eliminating both
cloud connectivity and local network access points that deployed devices rely
on, making support systems for these scenarios unavailable or non-operational.

A pragmatic issue in the context of smart homes is that without cloud
connectivity, users cannot interact with their appliances, such that during a
Wi-Fi outage smart lights become uncontrollable despite all hardware being
physically present. While merely inconvenient domestically, this architectural
dependency has critical implications in other contexts and \gls{ubabel} aims at
addressing such scenarios across different application domains: from
residential systems requiring local control, to building monitoring
infrastructures that must operate during emergencies, to disaster response
networks where autonomous operation becomes essential when traditional
infrastructure fails.

While cloud platforms like Amazon Alexa, Google Home, and Microsoft Azure \gls{iot}
Hub offer convenience for data processing and remote device control, their
inherent dependence on continuous Internet connectivity introduces some
critical limitations: increased latency from round-trip communication to
distant servers~\cite{8664595,8089336}, reduced availability during network
disruptions or cloud outages~\cite{9089244,maciel2022survey}, and poor
fault-tolerance when infrastructure fails~\cite{amiri2023resilient}.

These characteristics make cloud-centric architectures fundamentally unsuitable
for scenarios that require or prioritize local operation, autonomous behavior,
or guaranteed responsiveness during emergencies.

The Babel Ecosystem~\cite{fouto2022babel} partially addresses these limitations
by enabling devices to operate autonomously without cloud infrastructure, while
still supporting cloud integration when connectivity is available and desired.

In this document we propose \gls{ubabel}, a lightweight framework targeting
embedded platforms (e.g., ESP32, Raspberry Pi Pico) aimed at developing
resilient, multi-protocol, and decentralized \gls{iot} systems that can operate
autonomously during infrastructure failures.

\gls{ubabel} will integrate with the broader Babel Ecosystem, which runs on
hardware with greater resources such as full Raspberry Pi boards,
desktops/laptops, or servers. This enables a heterogeneous architecture
\todo{FIXED: "class-based"} where resource-constrained edge devices can
seamlessly interoperate with computational nodes for additional data
aggregation, processing, and large-scale coordination.

\paragraph{Note on terminology:} Throughout this work the term
\emph{multi-protocol} is used to refer to heterogeneity in communication
technologies (i.e., \gls{BLE}, \gls{LoRa}, Wi-Fi, etc.) that have distinct
link/physical layers, while \emph{multi-channel} refers to frequency diversity
within a single protocol (e.g., 2.4GHz Wi-Fi channels 1-13, \gls{LoRa}
frequency hopping). Multi-protocol operation provides technology diversity for
resilience; multi-channel operation provides frequency diversity for both
anti-jamming and additional throughput.

\section*{Main Research Questions}
\label{sec:research_questions}

In this work, we focus on three main research questions:

\paragraph{How can these systems maintain communication and full operation when
	traditional infrastructure fails?\\}

Conventional deployments of networked devices often heavily rely on Wi-Fi
access points, cellular towers, or other centralized infrastructure that can
easily become unavailable during disasters, or is altogether unreliable in
remote locations.

\gls{ubabel} will address this by leveraging a
multi-protocol communication approach, supporting a diverse set of wireless
protocols that can operate independently of infrastructure.

By enabling adaptive protocol selection and \gls{P2P} mesh formation, devices
can establish alternative communication paths when primary channels fail,
allowing them to sustain information exchange and system availability.

\todo{FIXED: muito ênfase em iot}

\paragraph{How can device heterogeneity be leveraged to create and orchestrate
	these systems?}

Deployments naturally comprise devices with varying capabilities, from simple
resource-constrained nodes, to more capable gateways that might act as
aggregators or bridges. Rather than treating this heterogeneity as a
limitation, \gls{ubabel} will exploit it through capability-aware protocols
that allow devices to negotiate roles dynamically via autonomous and automatic
discovery.

Resource-rich nodes can serve as data aggregation and processing points, or
bridges between a remote deployment and traditional network infrastructure (if
desired), while simpler devices focus on sensing and actuation, creating a
resilient multi-tier architecture.

\paragraph{How can we achieve (near-)zero-configuration deployment in diverse
environments?}

Straightforward deployments are essential where users cannot perform complex
configurations -- from smart homes and factories to hazardous areas.
\gls{ubabel} will provide automatic peer discovery across multiple protocols,
self-organizing network formation, and decentralized coordination mechanisms
that eliminate the need for pre-configured coordinator nodes or manual network
planning.
\todo{FIXED: demasiado foco em emergências}
Devices will autonomously establish connectivity with each other, negotiate
protocols, and begin operation upon being activated, with minimal configuration
effort to enable rapid deployment even in hard-to-reach locations.

% \todo{move to immediately before amazon azure etc, use as motivation like:
% "to motivate this consider these relevant examples of use cases: imagine that etc etc; that suffer from x y and z with cloud infra etc dependencies"}
% \todo{also the wifi goes out can't turn on house lights anedoctal example}
% \todo{a pragmatic issue in the context of smart home is that without connectivity to the cloud you can't interact with your home appliances etc, this can lead to more disatrous consequences in other contextes}
% TODOTODOTODO MOVE THIS TO LATER SECTIONS IN DETAIL
% \section{Use Case: Smart Classroom System}
%
% To apply these research questions in a concrete application, we plan to develop a 
% smart classroom monitoring system that operates in both normal and emergency 
% conditions.
%
% Under normal operation, the system displays useful daily information such as
% room schedules and occupancy, as well as noncritical environmental factors like
% temperature, humidity and  CO$_2$ levels; simultaneously, it coordinates
% with other devices in the ecosystem to detect potentially hazardous conditions,
% such as fires, structural instability or critical network failures.
%
% During such disasters, the same system automatically transitions to emergency
% mode, in which devices form peer-to-peer mesh networks for disseminating safety
% information and evacuation routes, perform real-time hazard mapping
% (smoke/flame detection, temperature monitoring), and adapt evacuation guidance
% based on blocked or presumed-dangerous routes.
%
% This dual-mode operation demonstrates how \gls{ubabel} systems can
% provide everyday utility while maintaining resilience and usefulness for
% emergency scenarios, without requiring infrastructure reconfiguration or manual
% intervention.

\section*{Expected Contributions}

We plan to make the following main contributions:

\begin{itemize}
    \item A decentralized architecture supporting multiple communication
        protocols (Wi-Fi, \gls{BLE}, \gls{LoRa}, \gls{ZigBee}, \gls{ESP-NOW})
        with adaptive switching based on \gls{QoS} requirements, resource
        availability and device capabilities;
    \item A resource-efficient programming framework for embedded platforms
        that enables autonomous operation without central coordination,
        providing abstractions for multi-protocol communication, peer
        discovery, and opportunistic data forwarding;
    \item A proof-of-concept implementation demonstrating infrastructure-
        independent operation and automatic disaster-mode failover in a
        real-world deployment.
\end{itemize}

\section*{Document Structure (Roadmap)}

\todo{FIXED: added doc structure}
The remainder of this document is arranged as follows:

\begin{description}
    \item[\textbf{Chapter~\ref{cha:related_work}}] Reviews widely-used
        communication protocols and related work in \gls{P2P} networking,
        multi-protocol communication, decentralized synchronization and data
        compression.
        % , and security.

	\item[\textbf{Chapter~\ref{cha:solution_arch}}] Presents the system
	      requirements, operating conditions, failure modes, hardware and
	      software platforms, and the architectural model of the proposed
	      solution.

    \item[\textbf{Chapter~\ref{cha:planning_progress}}] Summarizes the current
        implementation status and outlines the development and
        evaluation plan.
\end{description}

\todo{might be worth voltar atrás e resumir algo para não ter esta página neste estado}
