%!TEX root = ../template.tex
%%%%%%%%%%%%%%%%%%%%%%%%%%%%%%%%%%%%%%%%%%%%%%%%%%%%%%%%%%%%%%%%%%%
%% chapter1.tex
%% NOVA thesis document file
%%
%% Chapter with introduction
%%%%%%%%%%%%%%%%%%%%%%%%%%%%%%%%%%%%%%%%%%%%%%%%%%%%%%%%%%%%%%%%%%%

\typeout{NT FILE chapter1.tex}%

\chapter{Introduction}
\label{cha:introduction}

\prependtographicspath{{5-Figures/Covers/}}

% % epigraph configuration
% \epigraphfontsize{\small\itshape}
% \setlength\epigraphwidth{12.5cm}
% \setlength\epigraphrule{0pt}
%
% \includegraphics[width=0.1\linewidth]{NOVAthesisFiles/Images/novathesis-insignia}\hfill
% \includegraphics[width=0.875\linewidth]{NOVAthesisFiles/Images/novathesis-text}
%
% \noindent This is the \gls{novathesis} \LaTeX\ template \ntindex[Template!]{Version} \novathesisversion\ from   {Template!date}\novathesisdate.
%
% \epigraph{
%   This work is licensed under the \href{https://www.latex-project.org/lppl/lppl-1-3c/}{\LaTeX\ Project Public License v1.3c}.
%   To view a copy of this \ntindex[Template!]{license}, visit the \href{https://www.latex-project.org/lppl/}{LaTeX project public license}.
% }

% \section{Welcome to the \novathesis\ Template}
% \label{sec:if_you_use_this_template}

The proliferation of Internet of Things (IoT) devices has changed how we
monitor and interact with physical spaces, from smart homes to industrial
facilities.
However, current IoT systems remain fundamentally dependent on continuous
Internet connectivity, and often rely on centralized cloud infrastructures.
These factors lead to shortcomings in key areas that are in dire need of
addressing in the current technological landscape:\\

\begin{itemize}
    \item \textbf{Limited suitability for hazardous environments} — Remote or
        dangerous locations (industrial sites, disaster-prone areas) require
        systems that can operate reliably without constant human intervention
        or stable network infrastructure;
    \item \textbf{Lack of autonomous operation} — Device deployment and
        operation en masse can be brittle, with little tolerance for individual
        node failures in the IoT infrastructure, which is naturally susceptible
        to network failures and intermittent connectivity;
    \item \textbf{Privacy and data sovereignty concerns} — Cloud platforms and
        other third-parties are oftentimes an unavoidable middle layer between
        end devices and end users, raising questions about data processing and
        control.
\end{itemize}

These limitations are accentuated in disaster response and awareness scenarios,
where real-time data collection and dissemination are crucial. During
earthquakes, floods, and other emergencies, traditional communication
infrastructure often fails first, yet the need for sensor data (structural
integrity, air quality, evacuation routes) and bidirectional communication
(threats alerts, user feedback) becomes critical.

% \todo{mention Babel as alternative to azure, amazon, etc?
% 	in particular explain the cloud limitations and solutions to cloud, we're not 100\% against cloud, but like 60\% due to trust etc}
% \todo{iot is more focused on data collection and domotics more so on other things,
% 	iot cloud platforms usually focus on processing, and domotics has the APIs that allow remote device control}

While cloud platforms like Amazon Alexa, Google Home, and Microsoft Azure IoT
Hub offer convenience for data processing and remote device control, their
inherent dependence on continuous Internet connectivity makes them unsuitable
for scenarios that require or prioritize local operation\todo{mention latency,
availability and fault-tolerance here too?}. The Babel Ecosystem \todo{cite
Babel here!!} addresses these limitations by enabling devices to operate
autonomously without cloud infrastructure, while still supporting cloud
integration when connectivity is available and desired.

In this document we introduce \textbf{MicroBabel}, a lightweight framework targeting
embedded platforms (ESP32, Raspberry Pi Pico) aimed at developing resilient,
multi-channel and decentralized IoT systems that can operate autonomously
during infrastructure failures. \textbf{MicroBabel} integrates with the broader Babel
Ecosystem, which runs on more capable hardware such as full Raspberry Pi boards
or computers, enabling a tiered architecture where resource-constrained edge
devices can seamlessly interoperate with computational nodes for data
aggregation, processing, and coordination.

\section{Research Topics}
\label{sec:research_topics}

We focus on three main research questions:

\subsection{How can these systems maintain communication when traditional
infrastructure fails?}

Traditional IoT deployments rely heavily on Wi-Fi access points, cellular
towers, or other centralized infrastructure that often becomes unavailable
during disasters, or is altogether unreliable in remote locations.
\textbf{MicroBabel} addresses this by leveraging a multi-channel communication
approach, supporting a diverse protocol stack (BLE, LoRa, ESP-NOW\todo{maybe
not include esp-now since the picos don't support it?}, ZigBee, infrared) that
can operate independently of infrastructure.

By enabling adaptive protocol selection and peer-to-peer mesh formation,
devices can establish alternative communication paths when primary channels
fail.

\subsection{How can device heterogeneity be leveraged to create and orchestrate
these networks?}

IoT deployments naturally comprise devices with varying capabilities, from
resource-constrained sensors, to gateway nodes with greater processing power
and storage. Rather than treating this heterogeneity as a limitation,
\textbf{MicroBabel} exploits it through capability-aware protocols that allow
devices to negotiate roles dynamically via discovery.

Resource-rich nodes can serve as data aggregation and processing points, or
bridges between a remote deployment and??, while simpler devices focus on
sensing and actuation, creating a resilient multi-tier architecture.
\todo{is this too repetitive considering the paragraph before the research topics?}

\subsection{How can we achieve (near-)zero-configuration deployment, suitable for emergency scenarios and hazardous location monitoring where manual setup is impractical or even impossible?}

Emergency response and hazardous environment monitoring demand systems that 
can be deployed rapidly without extensive configuration. \textbf{MicroBabel} 
provides automatic peer discovery across multiple channels, self-organizing 
network formation, and decentralized coordination mechanisms that eliminate the 
need for pre-configured master nodes or manual network planning.

Devices autonomously establish connectivity with each other, negotiate protocols,
and begin operation upon being activated, enabling deployment by non-technical
personnel or in hard-to-reach locations.
\todo{is this too ambitious?}

\section{Use Case: Smart Classroom System}

To apply these research questions in a concrete application, we plan to develop a 
smart classroom monitoring system that operates in both normal and emergency 
conditions.

Under normal operation, the system displays useful daily information such as
room schedules and occupancy, as well as noncritical environmental factors like
temperature, humidity and  CO$_2$ levels; simultaneously, it coordinates
with other devices in the ecosystem to detect potentially hazardous conditions,
such as fires, structural instability or critical network failures.

During such disasters, the same system automatically transitions to emergency
mode, in which devices form peer-to-peer mesh networks for disseminating safety
information and evacuation routes, perform real-time hazard mapping
(smoke/flame detection, temperature monitoring), and adapt evacuation guidance
based on blocked or presumed-dangerous routes.

This dual-mode operation demonstrates how \textbf{MicroBabel} systems can
provide everyday utility while maintaining resilience and usefulness for
emergency scenarios, without requiring infrastructure reconfiguration or manual
intervention.

\section{Contributions}

We plan to make the following contributions:
\begin{itemize}
	\item A decentralized architecture supporting multiple communication
	      channels (Wi-Fi, Bluetooth Low Energy (BLE), LoRa, ZigBee) with
	      adaptive protocol switching based on Quality of Service (QoS)
	      requirements, resource availability and device capabilities;
	\item A resource-efficient programming framework for embedded platforms 
	      that enables autonomous operation without central coordination, 
	      providing abstractions for multi-channel communication, peer 
	      discovery, and opportunistic data forwarding;
	\item A proof-of-concept implementation demonstrating infrastructure-
	      independent operation and automatic disaster-mode failover in a 
	      real-world deployment.
\end{itemize}

\section{The \emph{NOVAthesis} Template}
\label{sec:a_bit_of_history}

\ntindex[Template]{}

The \gls{novathesis} template was born at the \gls{DI} of  \gls{FCT} of \gls{NOVA}, Portugal.  But the user base grew… initially grew to other Departments of FCT-NOVA, then to other Schools of NOVA, and later to other Schools of other Universities.  Currently more than~25 Schools are natively supported by the \gls{novathesis} template.  The full list of supported Schools can be founf at the
\href{https://github.com/joaomlourenco/novathesis}{Project's GitHub page}.

\subsection{Your Time is Precious}
\label{sub:time_is_money}

Did you learn how to drive by sitting by the wheel and throwing your car into the road?  Most probably you did take your time \emph{learning the rules} and \emph{practicing} first! Likewise, it is not wise to throw yourself at the task of writing a thesis/dissertation in \LaTeX\ without seriously considering the following \ntindex{recommendation}!

\begin{tcolorbox}[colback=green!8]
	If you are going to spend zillions of hours writing your thesis/dissertation using the \gls{novathesis} \LaTeX\ template (or some other \LaTeX\ template), be wise and spend a couple of hours learning how to use it properly by reading its manual.  And then, be even wiser, and spend a few more hours \href{https://github.com/joaomlourenco/novathesis/wiki\#learning-latex}{learning some \LaTeX}.  I am sure that the time you are investing now will pay itself countless times before you submit your thesis/dissertation.\\\parbox{\linewidth}{\raggedleft---~\emph{João Lourenço}}
\end{tcolorbox}

\subsection{Recognition}
\label{sub:recognition}

\ntindex[Recognition]{}

The \gls{novathesis} template was born in~1996, and what you see now accumulates to many many hundreds (thousands?!) of working hours, unpaid and stolen from family and friends.  This work is available to the community under the \href{LaTeX project public license}{\LaTeX\ Project Public License v1.3c}, which means you are entitled to use it for free and change it at your will.  However, if you decide to use this template to write your thesis/dissertation, \textbf{be fair to the developers} and:
\begin{enumerate}
	\item \ntindex[novathesis!Citation]{} Cite the \gls{novathesis} manual~\cite{novathesis-manual} in a place of your choice (e.g., in the \emph{Acknowledgments}) of your thesis/dissertation with “\verb!\cite{novathesis-manual}!” .  If you cite it this way, the correct entry will be added automatically to your bibliography (no need to worry with the necessary BibTeX entry, as it will be added automatically);
	\item Go to the
	      \href{https://github.com/joaomlourenco/novathesis}{\ntindex[GitHub!project web page]{project web page} in GitHub} and give the project a \ntindex[GitHub!stars]{star} (marked with a red ellipse at the top-right in Figure~\ref{fig:github}); and
	\item Make a \ntindex[donations]{donation} by visiting the \gls{novathesis} project page and clicking in the button marked with a green ellipse at the top-center in Figure~\ref{fig:github}).  Alternatively, just click \href{https://www.paypal.com/donate/?hosted_button_id=8WA8FRVMB78W8}{\fcolorbox{DarkGreen}{gray!15}{\textbf{~HERE~}}} and your browser will be directed to the right page.
\end{enumerate}

\begin{figure}[htbp]
	\centering
	\includegraphics[width=0.5\linewidth]{github1}
	\caption{The \gls{novathesis} project web page in GitHub.}
	\label{fig:github}
\end{figure}


\section{Getting Started}
\label{sec:getting_started}

The template provides an \emph{easy to use} setting for you to write your thesis/dissertation in \LaTeX:
\begin{itemize}
	\item  Select your school;
	\item Fill your thesis metadata (title, research field, etc) in the file “\texttt{template.tex}”;
	\item Create your thesis/dissertation contents using the files in folder “\texttt{Chapters}”; and
	\item Process using you favorite \LaTeX\ processor (pdf\LaTeX, \XeLaTeX\ or \LuaLaTeX).
\end{itemize}

\subsection{Using Overleaf}
\label{sub:using_overleaf}

\ntindex[Installation!Overleaf]{}
\ntindex[Using!Overleaf]{}

\newcommand{\Overleaf}{\href{https://www.overleaf.com?r=f5160636&rm=d&rs=b}{Overleaf}}

\begin{wrapfigure}{r}{0.3\linewidth}
	% \vspace*{-10ex}
	\includegraphics[width=\linewidth]{overleaf}%
	\caption{NOVAthesis template in Overleaf.}
	\label{fig:overleaf}
\end{wrapfigure}
\mbox{}\Overleaf\ is a collaborative cloud-based LaTeX editor used for writing, editing and publishing scientific documents. Like “Google Docs”,  for \LaTeX\ users. You can edit and compile your \LaTeX\ source on the cloud, without installing software in your own computer, and, much like \emph{Google Docs}, you can share your document with others users and everybody can edit the same file at the same time (this may be dangerous).

If you do not have an account in \Overleaf, you must \href{https://www.overleaf.com?r=f5160636&rm=d&rs=b}{create one first}.

Once you have an account, please access the \gls{novathesis} template in \href{https://www.overleaf.com/latex/templates/novathesis-v7-dot-1-18/jhqwhtcwbmqc}{Overleaf} and select the green button \emph{Open as Template} (see \Autoref{fig:overleaf}).

\bgroup
\itshape
Please notice that the version currently available in Overleaf (v7.1.18) is slightly outdated (current version is v\novathesisversion). A new version (v7.1.29) will be submitted to Overleaf soon.  Until then, please:
\begin{enumerate}
	\item Download the \href{https://github.com/joaomlourenco/novathesis/archive/main.zip}{latest version} from the GitHub repository as a Zip file.
	\item Login to your favorite LaTeX cloud service. I recommend \href{https://www.overleaf.com/?r=f5160636&rm=d&rs=b}{Overleaf} but there are alternatives (these instructions apply to Overleaf and you'll have to adapt for other providers).
	\item In the menu select: \texttt{New project} $\rightarrow$ \texttt{Upload project}.
	\item Upload the zip file.
	\item Select “template.tex” as the main file.
	\item Let Overleaf compile the document.
\end{enumerate}
\egroup

\begin{tcolorbox}[colback=red!8]
	Notice that you need a (student) subscription to compile the \novathesis\ template in Overleaf, otherwise your compilation will always time out.
\end{tcolorbox}

\subsection{Using a Local \LaTeX\ Installation Local}
\label{sub:using_local_latex}

\ntindex[Installation!Local installation]{}
\ntindex[Using!Local installation]{}

\begin{wrapfigure}{r}{0.3\linewidth}
	\vspace*{-7ex}
	\includegraphics[width=\linewidth]{github}%
	\caption{The NOVAthesis Project page in GitHub.}
	\label{fig:github2}
\end{wrapfigure}

First of all, start by installing \LaTeX\ in your computer.  There are two main distributions, \href{https://miktex.org}{\ntindex{\MikTeX}}\ and \href{https://www.tug.org/texlive/}{\ntindex{\TeXLive}}, and both of them are available for the~3 most popular Operating Systems: Linux, macOS and Windows.

Be aware that a full installation of \MikTeX\ or \TeXLive\ will take near~5\,GB of hard disk space.  So, think twice before installing the full distribution.  See the \gls{novathesis} Wiki for the \href{https://github.com/joaomlourenco/novathesis/wiki/installing-latex#minimal-installation-in-any-of-the-systems-above}{list of packages required to compile the template}.

Once you have \LaTeX\ up and running, remember to install a good \LaTeX\ text editor.  I recommend you to take a look at  \href{https://tex.stackexchange.com/questions/339/latex-editors-ides}{this post} in the \url{tex.stackexchange.com} site.  If you want a quick and dirty recommendation, try \href{https://www.texstudio.org/}{\ntindex{TeXStudio}}.

Now, you must access the \gls{novathesis} repository in \href{https://github.com/joaomlourenco/novathesis}{GitHub}, select the green button \emph{Code} and then \emph{download} (or \emph{clone}) the template.  You will always get the latest version of the template (currently v\novathesisversion\ from \novathesisdate).


\section{Getting Help}
\label{sec:getting_help}

\ntindex[Help]{}

No! You don't have to use this template to write your thesis.  You don't even have to use \LaTeX.  However, writing a thesis is serious stuff, and which tool you shall use to write it is not a decision to make lighthearted.

\LaTeX\ is hard enough by itself.  This template aims at making your life easier, but not easy. If you choose to use this template to write your thesis, you are very welcome.  However, don't expect me to provide you help with \LaTeX.  Look for help with your friends (you have some friends, don't you?), or search the web, or try even to read some book(s) on \LaTeX. In the end you will certainly find the experience rewarding.

When you come to the point of “\emph{How do I do this with the \novathesis\ template?}”, remember…

\begin{enumerate}
	\item To check the \href{https://github.com/joaomlourenco/novathesis/wiki}{\gls{novathesis} wiki} and have some hope!  \emojiSmile
	\item \href{https://www.google.com}{Google} is your best friend.
	\item Search the \href{https://github.com/joaomlourenco/novathesis/discussions}{GitHub Discussions page} for a question related to yours.  \emph{If and only if} you don't find one, then post your own question in English please!
	\item Search the \href{https://www.facebook.com/groups/novathesis}{NOVAtheis Facebook group} for a question related to yours.  \emph{If and only if} you don't find one, then post your own question in either Portuguese or English, at your preference.
\end{enumerate}

When you post your own question, remember to \textbf{always} state the \gls{novathesis} version number you are using and referring to.

\begin{tcolorbox}[colback=blue!8]
	\centering
	Please do not attempt to contact me directly (email, Messenger, etc)…\\I WILL NOT REPLY!
\end{tcolorbox}


\subsection{Suggestions, Bugs and Feature Requests} % (fold)
\label{sub:suggestions_bugs_and_feature_requests}

\begin{description}
	\item[Help:] If you just need some help, see above \Autoref{sec:getting_help}.
	\item[Suggestion:] \ntindex[Suggestions]{} Do you have a suggestion/recommendation? Please add it to the wiki and help other users!
	\item[Bug:] \ntindex[Bugs]{} Did you find a bug? Please open an issue. Thanks!
	\item[New Feature:] \ntindex[Feature Requests]{} Would you like to request a new feature (or support of a new School)? Please open an issue. Thanks!

\end{description}



% subsection suggestions_bugs_and_feature_requests (end)




\section{Donors}
\label{sec:donations}

\ntindex[Donations]{}

The \href{https://github.com/joaomlourenco/novathesis/wiki#donators}{list of \emph{Donnors}} is available in the \gls{novathesis} Project page.


\section{Disclaimer}
\label{sec:disclaimer}

\ntindex[Disclaimer]{}

Although the \gls{novathesis} template is endorsed by some Schools (e.g., \href{https://www.fct.unl.pt/estudante/informacao-academica/teses-e-dissertacoes}{linked from FCT-NOVA web site}), the \gls{novathesis} template \textbf{this not an official template} for any School.

The \gls{novathesis} template exists to make your life easier and we do our best to make it compliant to the supported ($+25$) Schools' regulations but, in the end of the line, you and only you are accountable for both the look and the contents of the document you submit as your thesis/dissertation.
