%!TEX root = ../template.tex
%%%%%%%%%%%%%%%%%%%%%%%%%%%%%%%%%%%%%%%%%%%%%%%%%%%%%%%%%%%%%%%%%%%
%% chapter1.tex
%% NOVA thesis document file
%%
%% Chapter with introduction
%%%%%%%%%%%%%%%%%%%%%%%%%%%%%%%%%%%%%%%%%%%%%%%%%%%%%%%%%%%%%%%%%%%

\typeout{NT FILE chapter1.tex}%

\chapter{Introduction}
\label{cha:introduction}

\prependtographicspath{{5-Figures/Covers/}}

The proliferation of Internet of Things (IoT) devices has changed how we
monitor and interact with physical spaces -- from smart homes to industrial
facilities -- with the global IoT market reaching 18.5 billion connected
devices in 2024, and projected to reach 39 billion by
2030\cite{sinha2025stateofiot}, with industry analysts estimating that IoT
technologies could unlock between \$5.5 and \$12.6 trillion in economic value
by the same year\cite{mckinsey2021iot}.

However, current IoT systems remain fundamentally dependent on continuous
Internet connectivity and centralized cloud
infrastructures\cite{8123913,s20226441}. The dominant architectural pattern
across the industry involves edge devices collecting data and transmitting it
to remote cloud servers for processing, storage, and other control logic. This
approach delivers scalability and ease of management but creates a fundamental
dependency on network availability.

This centralized model introduces several concerns beyond just connectivity:
operational costs for cloud services create economic dependencies on
third-party providers and often lead to vendor lock-in, data sovereignty issues
arise when sensitive information must travel to and reside on external
infrastructure, and system resilience becomes fundamentally tied to the
availability of these remote services.

While cloud platforms can benefit IoT deployments through additional
computational resources and storage capacity, systems that \emph{depend} on
constant cloud connectivity sacrifice local autonomy and introduce single
points of failure. This cloud-centric model has enabled rapid IoT adoption, but
introduces critical vulnerabilities across scenarios where continuous
connectivity cannot be guaranteed.

Recent infrastructure failures illustrate the fragility of cloud-dependent
architectures: the AWS US-EAST-1 outage in October
2025\cite{thousandeyes2025aws} disrupted services from banking to smart homes
worldwide, demonstrating how a DNS configuration error in a \emph{single
region} can cascade into global disruptions\cite{aws2025summary}.

Attempts have been made to address these challenges through incremental
improvements, such as shifting focus to edge computing in order to reduce
latency\cite{s20226441,8123913,8664595}, redundant cloud regions for
availability\cite{mesbahi2018reliability,maciel2022survey}, and hybrid
architectures that combine local and remote
processing\cite{9846938,s24165320,KREKOVIC2025101553}.

However, these solutions remain fundamentally tied to the assumption of
eventual connectivity, and often increase system complexity without eliminating
the core dependency. This leads to shortcomings in key areas that demand
more fundamental architectural reconsideration:

\begin{description}
    \item \textbf{Limited suitability for hazardous environments}\\
        Remote or dangerous locations (industrial sites, disaster-prone areas)
        require systems that can operate reliably without constant human
        intervention or stable network infrastructure;
    \item \textbf{Lack of autonomous operation}\\
        Device deployment and operation en masse can be brittle, with little
        tolerance for individual node failures in the IoT infrastructure, which
        is naturally susceptible to network failures and intermittent
        connectivity;
    \item \textbf{Privacy and data sovereignty concerns}\\
        Cloud platforms and other third-parties are oftentimes an unavoidable
        middle layer between end devices and end users, raising questions about
        data processing and control.
\end{description}

\todo{fazer distinção maior entre iot e domotics?}
While these challenges affect both IoT and domotics systems alike -- from
industrial monitoring to residential automation -- they become particularly
acute in disaster response and emergency scenarios, where communication
infrastructure fails precisely when needed most, rapid deployment with minimal
configuration becomes essential, and autonomous operation transitions from
desirable to indispensable.

During earthquakes, floods, and other emergencies, the need for real-time
sensor data (structural integrity, air quality, evacuation routes) and
bidirectional communication (threat alerts, user feedback) becomes critical,
yet traditional infrastructure often fails first, eliminating both cloud
connectivity and local network access points that deployed devices rely on.

A pragmatic issue in the context of smart homes is that without cloud
connectivity, users cannot interact with their appliances, such that during a
Wi-Fi outage smart lights become uncontrollable despite all hardware being
physically present.
While merely inconvenient domestically, this architectural dependency has
critical implications in other contexts and \textbf{MicroBabel} addresses such
scenarios across scales: from residential systems requiring local control, to
building monitoring infrastructures that must operate during emergencies, to
disaster response networks where autonomous operation becomes essential when
traditional infrastructure fails.

While cloud platforms like Amazon Alexa, Google Home, and Microsoft Azure IoT
Hub offer convenience for data processing and remote device control, their
inherent dependence on continuous Internet connectivity introduces some
critical limitations: increased latency from round-trip communication to
distant servers\cite{8664595,8089336}, reduced availability during network
disruptions or cloud outages\cite{9089244,maciel2022survey}, and poor
fault-tolerance when infrastructure fails\cite{amiri2023resilient}.

These characteristics make cloud-centric architectures fundamentally unsuitable
for scenarios that require or prioritize local operation, autonomous behavior,
or guaranteed responsiveness during emergencies.

The Babel Ecosystem~\cite{fouto2022babel} addresses these limitations by
enabling devices to operate autonomously without cloud infrastructure, while
still supporting cloud integration when connectivity is available and desired.

In this document we introduce \textbf{MicroBabel}, a lightweight framework
targeting embedded platforms (ESP32, Raspberry Pi Pico) aimed at developing
resilient, multi-protocol and decentralized IoT systems that can operate
autonomously during infrastructure failures. \textbf{MicroBabel} integrates
with the broader Babel Ecosystem, which runs hardware with greater resources
such as full Raspberry Pi boards or computers, enabling a tiered architecture
where resource-constrained edge devices can seamlessly interoperate with
computational nodes for data aggregation, processing, and coordination.

\section*{Research Topics}
\label{sec:research_topics}

We focus on three main research questions:

\paragraph{How can these systems maintain communication when traditional
infrastructure fails?}

Traditional IoT deployments rely heavily on Wi-Fi access points, cellular
towers, or other centralized infrastructure that often becomes unavailable
during disasters, or is altogether unreliable in remote locations.
\textbf{MicroBabel} addresses this by leveraging a multi-protocol communication
approach, supporting a diverse protocol stack (Bluetooth Low Energy (BLE),
LoRa, ESP-NOW, ZigBee, infrared) that can operate independently of
infrastructure.

By enabling adaptive protocol selection and peer-to-peer mesh formation,
devices can establish alternative communication paths when primary channels
fail.

\paragraph{How can device heterogeneity be leveraged to create and orchestrate
these networks?}

IoT deployments naturally comprise devices with varying capabilities, from
resource-constrained sensors, to gateway nodes with greater processing power
and storage. Rather than treating this heterogeneity as a limitation,
\textbf{MicroBabel} exploits it through capability-aware protocols that allow
devices to negotiate roles dynamically via discovery.

Resource-rich nodes can serve as data aggregation and processing points, or
bridges between a remote deployment and traditional network infrastructure (if
desired), while simpler devices focus on sensing and actuation, creating a
resilient multi-tier architecture.

\paragraph{How can we achieve (near-)zero-configuration deployment for emergency scenarios?}
% \paragraph{How can we achieve (near-)zero-configuration deployment, suitable for emergency scenarios and hazardous location monitoring where manual setup is impractical or even impossible?}

Emergency response and hazardous environment monitoring demand systems that 
can be deployed rapidly without extensive configuration. \textbf{MicroBabel} 
provides automatic peer discovery across multiple protocols, self-organizing 
network formation, and decentralized coordination mechanisms that eliminate the 
need for pre-configured master nodes or manual network planning.

Devices autonomously establish connectivity with each other, negotiate protocols,
and begin operation upon being activated, enabling deployment by non-technical
personnel or in hard-to-reach locations.
\todo{fico na dúvida se devia reword this para não "prometer too much"?}

% \todo{move to immediately before amazon azure etc, use as motivation like:
% "to motivate this consider these relevant examples of use cases: imagine that etc etc; that suffer from x y and z with cloud infra etc dependencies"}
% \todo{also the wifi goes out can't turn on house lights anedoctal example}
% \todo{a pragmatic issue in the context of smart home is that without connectivity to the cloud you can't interact with your home appliances etc, this can lead to more disatrous consequences in other contextes}
% TODOTODOTODO MOVE THIS TO LATER SECTIONS IN DETAIL
% \section{Use Case: Smart Classroom System}
%
% To apply these research questions in a concrete application, we plan to develop a 
% smart classroom monitoring system that operates in both normal and emergency 
% conditions.
%
% Under normal operation, the system displays useful daily information such as
% room schedules and occupancy, as well as noncritical environmental factors like
% temperature, humidity and  CO$_2$ levels; simultaneously, it coordinates
% with other devices in the ecosystem to detect potentially hazardous conditions,
% such as fires, structural instability or critical network failures.
%
% During such disasters, the same system automatically transitions to emergency
% mode, in which devices form peer-to-peer mesh networks for disseminating safety
% information and evacuation routes, perform real-time hazard mapping
% (smoke/flame detection, temperature monitoring), and adapt evacuation guidance
% based on blocked or presumed-dangerous routes.
%
% This dual-mode operation demonstrates how \textbf{MicroBabel} systems can
% provide everyday utility while maintaining resilience and usefulness for
% emergency scenarios, without requiring infrastructure reconfiguration or manual
% intervention.

\section*{Contributions}

We plan to make the following contributions:
\begin{itemize}
	\item A decentralized architecture supporting multiple communication
	      channels (BLE, LoRa, ZigBee, ESP-NOW, IR) with
	      adaptive protocol switching based on Quality of Service (QoS)
	      requirements, resource availability and device capabilities;
	\item A resource-efficient programming framework for embedded platforms 
	      that enables autonomous operation without central coordination, 
	      providing abstractions for multi-protocol communication, peer 
	      discovery, and opportunistic data forwarding;
	\item A proof-of-concept implementation demonstrating infrastructure-
	      independent operation and automatic disaster-mode failover in a 
	      real-world deployment.
\end{itemize}
