%!TEX root = ../../template.tex

\section{Planning}

\subsection{Validation}
\label{subsec:validation}

To validate \gls{ubabel}'s implementation, we intend to follow a three-tier
approach with individual component verification, integration testing, and
deployment evaluation, measuring for different metrics at each turn.

\paragraph{Component Verification:} (1) Message delivery rates, latency,  and
energy consumption per radio communication protocol (\textbf{communication
manager}); (2) Discovery time and neighbor view consistency across radio
combinations (\textbf{peer discovery}); (3) Active/passive view maintenance and
failure detection latency (\textbf{topology manager}); (4) Clock drift
compensation, gossip convergence speed, and multi-radio sync overhead
(\textbf{synchronization service}).

\paragraph{Integration Testing:} (1) Network formation time without access
points or coordinators (\textbf{infrastructure independence}); (2) Switching
latency and stability under changing conditions (\textbf{radio protocol
adaptation}); (3) Recovery time after node failures, message delivery during
network partitions (\textbf{fault tolerance}); (4) Performance degradation as
network size increases (\textbf{scalability}); (5) Message delivery times
across multi-hop paths with varying radio combinations (\textbf{end-to-end
latency});

\subsection{Use Cases}
\label{subsec:use_cases}

To guide development and further validate \gls{ubabel}'s implementation, we
consider two complementary scenarios that showcase different aspects of the
system's capabilities. Both use cases adhere to a general hierarchy:

\begin{description}
    \item[Sensing Tier:] Devices performing direct environmental or physical
        monitoring, typically using short-range wireless technologies for
        communication among themselves.
	
    \item[Actuation Tier:] Devices providing physical feedback to users or the
        environment (i.e., displays, alarms, controls), receiving commands from
        higher tiers.
	
    \item[Aggregation Tier:] Devices responsible zones or regions, aggregating
        data from the sensing tier, controlling the actuation tier, and bridging
        autonomous regions for wide-spread communication.
	
	\item[Coordination Tier:] System-wide coordination and optional external
	      integration (cloud services, building management systems), typically
	      requiring more capable hardware.
\end{description}

These following scenarios are meant to demonstrate operation during normal and
degraded conditions, taking advantage of a heterogeneous collection of devices
with optional cloud integration when available.

\subsubsection*{Campus-Wide Environmental Monitoring}
\label{subsubsec:uc_campus}

\paragraph{Deployment Context:}
Sensors distributed across multiple buildings in a university campus to monitor
environmental conditions (e.g., temperature, humidity, CO$_2$ levels,
occupancy) and classroom occupancy:\todo{TODO: look into Millimeter Wave Radar}

\begin{itemize}
    \item \textbf{Sensing:} Battery-powered ESP32/Pico 2 W devices (Class-1)
        throughout rooms and corridors
    \item \textbf{Actuation:} Small displays showing current conditions and
        occupancy status; battery-power as backup if using low-power e-paper
        technology
    \item \textbf{Aggregation:} ESP32-S3/P4 devices (Class-1) as floor/building
        gateways using \gls{BLE}/\gls{ESP-NOW} locally, and \gls{LoRa} for
        inter-building communication; larger batteries supply this tier once
        grid power becomes unavailable
    \item \textbf{Coordination:} Grid-powered Raspberry Pi devices (Class-2) in
        server rooms, providing \gls{MQTT} brokerage and cloud connectivity
\end{itemize}

\paragraph{Normal Operation Mode:} Full hierarchical connectivity: sensing tier
using \gls{BLE}/\gls{ESP-NOW} within rooms and corridors to gather
environmental information; actuation tier using the same technologies to
provide real-time data in information hubs; aggregation tier providing
inter-building communication via \gls{LoRa}; and coordination tier providing
cloud integration.

This mode validates:
(1) Hybrid wireless communication;
(2) Topology management;
(4) Optional cloud integration (\gls{NFR}, Section~\ref{subsubsec:nonfunc_reqs}).

\paragraph{Offline Operation Mode:} Upon infrastructure failure (simulated
power outage or network collapse), coordination tier becomes unavailable but
the other three continue autonomous operation through \gls{P2P} gossip,
with data buffered for later storage once regular connectivity is reinstated. 

This mode validates:
(1) Infrastructure independence (\gls{FR}, Section~\ref{subsubsec:func_reqs});
(2) Autonomous operation;
(3) Communication technology adaptation;
(4) Store-and-forward mechanisms;
(5) Graceful degradation.

\subsubsection*{Building Safety and Evacuation System}
\label{subsubsec:uc_safety}

\paragraph{Deployment Context:}
Safety monitoring system combining hazard detection (smoke, fire, CO$_2$) with
evacuation guidance:

\begin{itemize}
	\item \textbf{Sensing:} Class-1 hazard sensors at strategic locations
	      (stairwells, corridors, rooms)
	\item \textbf{Actuation:} Class-1 devices with e-paper displays showing
	      evacuation routes and hazard warnings
	\item \textbf{Aggregation:} Class-1 per-floor gateways maintaining floor-level
	      coordination and dynamically computing safe routes
	\item \textbf{Coordination:} Class-2 building coordinator integrating with
	      building management systems
\end{itemize}

\paragraph{Normal Operation Mode:} Continuous monitoring with sensing tier using
\gls{BLE}/\gls{ESP-NOW} to aggregation tier, which forwards to coordination
tier for logging and analysis.

This mode validates:
(1) Priority-based queuing (critical vs. periodic data);
(2) Low-latency local communication for real-time monitoring;
(3) Integration with existing infrastructure.

\paragraph{Disaster Operation Mode:} Upon hazard detection or infrastructure
failure, sensing tier broadcasts alerts via available wireless channels,
aggregation tier autonomously computes evacuation routes 
\todo{layout info stored in an SD card? much to consider}
and updates actuation tier displays which show real-time guidance. Critical
functions remain operational without coordination tier.

This mode validates:
(1) Autonomous operation;
(2) Multi-radio redundancy for resilience;
(3) Decentralized coordination;
(4) Adaptive communication technology selections;
(5) Real-time responsiveness.

\subsection{Scheduling}

In order to achieve the proposed requirements
(Section~\ref{subsubsec:func_reqs}), we delineate a set of tasks to be carried
out in the coming months:

\begin{description}
    \item[Task 1 -- Developing Core Components:]  Protocol Manager, Event
        Dispatcher validation, Communication Manager, and Time Synchronization
        Service as the foundation of our framework.

    \item[Task 2 -- Developing User-Level Protocols:]Peer Discovery Service and
        Topology Manager for autonomous mesh operation.

    \item[Task 3 -- Validating the implementation:] Use case development
        parallel to Tasks 1 and 2, followed by experimental setup and
        performance evaluation.

    \item[Task 4 -- Completing the Dissertation:] Final document preparation
        for potential conference submission (PerCom 2027).
\end{description}

The GANTT chart in Figure~\ref{fig:gantt_chart} presents the timeline for
completing these tasks.

% trim=left bottom right top,
\begin{figure}[htbp]
  \centering
  \includegraphics[width=\linewidth, trim=1.5cm 11.6cm 2.9cm 1.9cm, clip]{2-MainMatter/chapter4_sections/GANTT.pdf}
  \caption{Proposed timeline}
  \label{fig:gantt_chart}
\end{figure}
