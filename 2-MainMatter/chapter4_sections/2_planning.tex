%!TEX root = ../../template.tex

\section{Planning}

\subsection{Validation}
\label{subsec:validation}

To validate \gls{ubabel}'s implementation, we intend to follow a three-tier
approach with individual component verification, integration testing, and
deployment evaluation, measuring for different metrics at each turn.

\paragraph{Component Verification:} (1) Message delivery rates, latency,  and
energy consumption per protocol (\textbf{communication manager}); (2) Discovery
time and neighbor view consistency across protocol combinations (\textbf{peer
discovery}); (3) Active/passive view maintenance and failure detection latency
(\textbf{topology manager}); (4) Clock drift compensation, gossip convergence
speed, and multi-protocol sync overhead (\textbf{synchronization service}); (5)
Compression ratios, prediction accuracy, and computational overhead
(\textbf{compression service}).

% \begin{itemize}
% 	\item \textbf{Protocol handlers:} Message delivery rates, latency bounds,
% 	      and energy consumption per protocol
% 	\item \textbf{Peer discovery:} Discovery time and neighbor view consistency
% 	      across protocol combinations
% 	\item \textbf{Topology manager:} Active/passive view maintenance and
% 	      failure detection latency
% 	\item \textbf{Synchronization service:} Clock drift compensation, gossip
% 	      convergence speed, and multi-protocol sync overhead
% 	\item \textbf{Compression service:} Compression ratios, prediction
% 	      accuracy, and computational overhead on Class-1 devices
% \end{itemize}

\paragraph{Integration Testing:} (1) Network formation time without access
points or coordinators (\textbf{infrastructure independence}); (2) Switching
latency and stability under changing conditions (\textbf{protocol adaptation});
(3) Recovery time after node failures, message delivery during network
partitions (\textbf{fault tolerance}); (4) Performance degradation as network
size increases (\textbf{scalability}); (5) Message delivery times across
multi-hop paths with varying protocol combinations (\textbf{end-to-end
latency});

% \begin{itemize}
% 	\item \textbf{Infrastructure independence:} Network formation time without
% 	      access points or coordinators
% 	\item \textbf{Protocol adaptation:} Switching latency and stability under
% 	      changing conditions (e.g., interference, protocol unavailability)
% 	\item \textbf{Fault tolerance:} Recovery time after node failures, message
% 	      delivery during network partitions
% 	\item \textbf{Scalability:} Performance degradation as network size
% 	      increases
% 	\item \textbf{End-to-end latency:} Message delivery times across multi-hop
% 	      paths with varying protocol combinations
% \end{itemize}

% \paragraph{Deployment Evaluation:}
% Real-world validation using the use cases from Section~\ref{subsec:use_cases}:
%
% \begin{itemize}
% 	\item \textbf{Campus monitoring scenario:} Multi-building deployment
% 	      measuring autonomous operation duration, data delivery rates, and
% 	      battery lifetime under normal vs. disaster modes
% 	\item \textbf{Building safety scenario:} Single-building deployment
% 	      measuring critical alert propagation, evacuation guidance system, and
% 	      responsiveness during simulated infrastructure failures
% \end{itemize}
%
% Where applicable, performance will be compared against centralized alternatives
% (e.g., MQTT based systems) to quantify the benefits and drawbacks of our
% approach.

\subsection{Use Cases}
\label{subsec:use_cases}

To guide development and further validate \gls{ubabel}'s implementation, we
consider two complementary scenarios that showcase different aspects of the
system's capabilities. These are meant to demonstrate operation during normal
and simulated disaster conditions, taking advantage of a heterogeneous
collection of devices with optional cloud integration.

\subsubsection*{Campus-Wide Environmental Monitoring}
\label{subsubsec:uc_campus}

\paragraph{Deployment Context:}

A university campus with sensors distributed across multiple buildings to
monitor environmental conditions (e.g., temperature, humidity, CO$_2$ levels,
occupancy) and classroom occupancy\todo{TODO: look into Millimeter Wave Radar}.
The deployment consists of:

\begin{itemize}
    \item \textbf{Class-1 sensor nodes:} Battery-powered devices (ESP32-family,
        Pico 2 W) placed throughout rooms and corridors, performing periodic
        sensing
    \item \textbf{Class-1 gateway nodes:} Battery-powered devices with greater
        resources (ESP32-S3/P4) at strategic locations within each building,
        serving as protocol bridges and data aggregators
    \item \textbf{Class-2 aggregation nodes:} Grid-powered Raspberry Pi devices
        in server rooms or management offices, providing additional processing
        capacity and optional cloud connectivity
\end{itemize}

\paragraph{Normal Operation Mode:} During stable conditions, the system
operates with cloud connectivity when available, with Class-2 devices acting as
\gls{MQTT} brokers and makeshift cloud servers. Sensor nodes form local meshes
within buildings, forwarding data through gateway nodes to building-level
aggregators. 

This mode validates:
(1) Multi-protocol communication for different ranges: \gls{BLE} within rooms,
\gls{ESP-NOW}/\gls{ZigBee} between floors, \gls{LoRa} across buildings;
(2) Topology management with hybrid active/passive views;
(3) Data reduction for efficient transmission;
(4) Optional cloud integration (\gls{NFR},
Section~\ref{subsubsec:nonfunc_reqs}).

% This mode validates:
% \begin{itemize}
%     \item Multi-protocol communication for different ranges: \gls{BLE} within
%         rooms, \gls{ESP-NOW}/\gls{ZigBee} between floors, \gls{LoRa} across
%         buildings
%     \item Topology management with hybrid active/passive views
%     \item Data reduction for efficient transmission
%     \item Optional cloud integration (Non-Functional Requirement,
%         Section~\ref{subsubsec:nonfunc_reqs})
% \end{itemize}

\paragraph{Disaster Operation Mode:} Upon infrastructure failure (simulated
power outage or network collapse), battery-powered sensors continue autonomous
operation. The system maintains local coordination through \gls{P2P} gossip
interactions, with Class-1 gateways aggregating data from their vicinity and
coordinating between buildings via \gls{LoRa}.

This mode validates:
(1) Infrastructure independence (\gls{FR}, Section~\ref{subsubsec:func_reqs});
(2) Autonomous operation without cloud connectivity;
(3) Protocol adaptation when primary mediums become unavailable;
(4) Store-and-forward mechanisms;
(5) Resource-efficient operation on limited battery power.

% This mode validates:
% \begin{itemize}
%     \item Infrastructure independence (Functional Requirement,
%         Section~\ref{subsubsec:func_reqs})
%     \item Autonomous operation without cloud connectivity
%     \item Protocol adaptation when primary mediums become unavailable
%     \item Store-and-forward mechanisms
%     \item Resource-efficient operation on limited battery power
% \end{itemize}

\subsubsection*{Building Safety and Evacuation System}
\label{subsubsec:uc_safety}

\paragraph{Deployment Context:}
A safety-focused monitoring system, combining environmental hazard detection
(smoke, fire, CO$_2$, structural integrity\todo{a "earthquake detection" de
SCMU não funcionava muito bem... maybe outra coisa?}) with occupancy tracking
and evacuation guidance. This use case expands upon the campus scenario by
emphasizing real-time responsiveness and critical message prioritization:

\begin{itemize}
    \item \textbf{Class-1 hazard sensors:} Fixed sensors at key locations
        (stairwells, corridors, rooms) for smoke, temperature, and air quality
        monitoring
    \item \textbf{Class-1 gateway nodes:} Per-floor devices maintaining
        floor-level mesh coordination and hazard aggregation
    \item \textbf{Class-2 building coordinator:} Central aggregation point
        integrating with existing building management systems when available
    \item \textbf{Mobile display nodes:} Battery-powered displays (ESP32-S3/P4
        with e-paper or small screens) showing evacuation routes and hazard
        warnings
\end{itemize}

\paragraph{Normal Operation Mode:} The system continuously monitors
environmental conditions and occupancy patterns, integrating with building
management infrastructure for logging and analysis. Sensor readings are
transmitted via short-range protocols (\gls{BLE}, \gls{ESP-NOW}) to floor
gateways, which forward aggregated data to the building coordinator.

This mode validates:
(1) Priority-based message queuing (critical vs. periodic data);
(2) Low-latency local communication for real-time monitoring;
(3) Integration with existing infrastructure.

% This mode validates:
% \begin{itemize}
%     \item Priority-based message queuing (critical vs. periodic data)
%     \item Low-latency local communication for real-time monitoring
%     \item Integration with existing infrastructure
% \end{itemize}

\paragraph{Disaster Operation Mode:} Upon detecting hazardous conditions or
infrastructure failure, the system automatically transitions to emergency mode.
Hazard sensors immediately broadcast critical alerts via available protocols,
floor gateways compute safe evacuation routes based on detected
hazards\todo{might be overpromising, cada um teria de manter some sort of
layout info...}, and displays show real-time evacuation guidance reflecting
current conditions.
\todo{ou.... layout info stored in an SD card? much to consider}

This mode validates:
(1) Autonomous operation during infrastructure collapse;
(2) Multi-protocol broadcast for critical alerts (redundancy through protocol
diversity);
(3) Decentralized coordination without central controllers;
(4) Adaptive protocol selection under emergency conditions;
(5) Real-time responsiveness despite limited resources.

% This mode validates:
% \begin{itemize}
%     \item Autonomous operation during infrastructure collapse (Functional
%         Requirement)
%     \item Multi-protocol broadcast for critical alerts (redundancy through
%         protocol diversity)
%     \item Decentralized coordination without central controllers
%     \item Adaptive protocol selection under emergency conditions
%     \item Real-time responsiveness despite limited resources
% \end{itemize}

\subsection{Task Scheduling}

In order to achieve the proposed requirements
(Section~\ref{subsubsec:func_reqs}), we delineate a set of tasks to be carried
out in the coming months:

\begin{description}
	\item[Task 1 -- Core Components:] Foundational protocol and communication
	      primitives
	      \begin{description}
		      \item[Task 1.1:] Protocol Manager (2 weeks)
		      \item[Task 1.2:] Event Dispatcher Validation (1 week)\todo{exagero?}
		      \item[Task 1.3:] Communication Manager (2 weeks)
	      \end{description}

	\item[Task 2 -- Discovery \& Topology:] Network formation, discovery and
	      routing mechanisms
	      \begin{description}
		      \item[Task 2.1:] Peer Discovery Service (2 weeks)
		      \item[Task 2.2:] Topology Manager (2 weeks)
	      \end{description}

	\item[Task 3 -- Coordination Layer:] Synchronization Service (2 weeks)

    \item[Task 4 -- Data Management]: Message queueing, lightweight reduction
        and compression
	      \begin{description}
		      \item[Task 4.1:] Message Queue Manager (2 weeks)
		      \item[Task 4.2:] Compression Service (2 weeks)
	      \end{description}

	\item[Task 5 -- Integration] \todo{I mean duh... se calhar retiro isto, era a pensar no integration hell ser chato e demorado de resolver}

	\item[Task 6 -- Validation:] Following the approach outlined in
	      Section~\ref{subsec:validation}
	      \begin{description}
		      \item[Task 6.1:] Use Case Development (2 weeks)
		      \item[Task 6.2:] Experimental Setup (1 week)
		      \item[Task 6.3:] Performance Evaluation (1 week)
	      \end{description}

	\item[Task 7 -- Dissertation Work:] Elaboration of the final document
	      \begin{description}
		      \item[Task 7.1:] Writing
		      \item[Task 7.2:] Preparation for PerCom (2027)\todo{perhaps too ambitious...?}
	      \end{description}
\end{description}

The GANTT
\todo{needs to be updated}
chart in Figure~\ref{fig:gantt_chart} presents the timeline
for completing these tasks.
