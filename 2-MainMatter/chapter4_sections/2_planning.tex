%!TEX root = ../../template.tex

\section{Planning}

\subsection{Validation}
\label{subsec:validation}

To validate \gls{ubabel}'s implementation, we plan to follow a three-tier
approach: component verification, integration testing, and deployment
evaluation.

\paragraph{Component Verification}
Individual components are validated in isolation to ensure correct behavior
under controlled conditions:

\begin{itemize}
	\item \textbf{Protocol handlers:} Message delivery rates, latency bounds,
	      and energy consumption per protocol
	\item \textbf{Peer discovery:} Discovery time and neighbor view consistency
	      across protocol combinations
	\item \textbf{Topology manager:} Active/passive view maintenance and
	      failure detection latency
	\item \textbf{Synchronization service:} Clock drift compensation, gossip
	      convergence speed, and multi-protocol sync overhead
	\item \textbf{Compression service:} Compression ratios, prediction
	      accuracy, and computational overhead on Class-1 devices
\end{itemize}

\paragraph{Integration Testing:}
System properties are evaluated through controlled network scenarios:

\begin{itemize}
	\item \textbf{Infrastructure independence:} Network formation time without
	      access points or coordinators
	\item \textbf{Protocol adaptation:} Switching latency and stability under
	      changing conditions (e.g., interference, protocol unavailability)
	\item \textbf{Fault tolerance:} Recovery time after node failures, message
	      delivery during network partitions
	\item \textbf{Scalability:} Performance degradation as network size
	      increases
	\item \textbf{End-to-end latency:} Message delivery times across multi-hop
	      paths with varying protocol combinations
\end{itemize}

\paragraph{Deployment Evaluation:}
Real-world validation using the use cases from Section~\ref{subsec:use_cases}:

\begin{itemize}
	\item \textbf{Campus monitoring scenario:} Multi-building deployment
	      measuring autonomous operation duration, data delivery rates, and
	      battery lifetime under normal vs. disaster modes
	\item \textbf{Building safety scenario:} Single-building deployment
	      measuring critical alert propagation, evacuation guidance system, and
	      responsiveness during simulated infrastructure failures
\end{itemize}

\paragraph{Baseline Comparison:} Where applicable, performance is compared
against centralized alternatives (e.g., MQTT based systems) to quantify
the benefits and drawbacks of our approach.

\subsection{Task Scheduling}

In order to achieve the proposed requirements
(Section~\ref{subsubsec:func_reqs}), we delineate a set of tasks to be carried
out in the coming months:

\todo{too verbose? compactar as tasks?}
\begin{description}
	\item[Task 1 -- Core Components:] Foundational protocol and communication
	      primitives
	      \begin{description}
		      \item[Task 1.1:] Protocol Manager
		      \item[Task 1.2:] Event Dispatcher Validation
		      \item[Task 1.3:] Communication Manager
	      \end{description}

	\item[Task 2 -- Discovery \& Topology:] Network formation, discovery and
	      routing mechanisms
	      \begin{description}
		      \item[Task 2.1:] Peer Discovery Service
		      \item[Task 2.2:] Topology Manager
		      \item[Task 2.3:] Routing Service
	      \end{description}

	\item[Task 3 -- Coordination Layer:] Coordination and Synchronization
	      mechanisms
	      \begin{description}
		      \item[Task 3.1:] Protocol Selector
		      \item[Task 3.2:] Synchronization Service
	      \end{description}

	\item[Task 4 -- Data Management]: Lightweight reduction and compression
	      strategies
	      \begin{description}
		      \item[Task 4.1:] Compression Service
		      \item[Task 4.2:] Message Queue
	      \end{description}

	\item[Task 5 -- Integration] \todo{I mean duh... se calhar retiro isto, era a pensar no integration hell ser chato e demorado de resolver}

	\item[Task 6 -- Validation:] Following the approach outlined in
	      Section~\ref{subsec:validation}
	      \begin{description}
		      \item[Task 6.1:] Use Case Development
		      \item[Task 6.2:] Experimental Setup
		      \item[Task 6.3:] Performance Evaluation
	      \end{description}

	\item[Task 7 -- Dissertation Work:] Elaboration of the final document
	      \begin{description}
		      \item[Task 7.1:] Writing
              \item[Task 7.2:] Preparation for PerCom (2027)\todo{perhaps too ambitious...?}
	      \end{description}
\end{description}

