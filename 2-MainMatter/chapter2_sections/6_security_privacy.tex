%!TEX root = ../../template.tex

\section{Security Considerations and Scope Delimitation}
\label{sec:security_scope}
\todo{FIXED: delimitar scope, still mention que olhei para security stuff}

This thesis focuses on resilience and decentralization at the physical and link
layers, and, as such, security mechanisms operating at higher layers (i.e.,
authentication, key management, privacy-preserving mechanisms) are considered
out of scope for the present body of work.

Security and privacy are nonetheless critical concerns in emergency and
disaster-focused scenarios, where sensitive data such as location tracking and
environmental conditions could be the target of malicious actors seeking to
disrupt system operation, inject false information, or compromise user safety.

Substantial research effort has been devoted to addressing these concerns in
resource-constrained \gls{iot} systems. While a comprehensive discussion of
such mechanisms is beyond the scope of this work, a brief overview of relevant
approaches is provided below to contextualize potential future extensions of
the proposed \gls{ubabel} framework.\\

Several lightweight authentication and key management approaches demonstrate
that cryptographic operations based on hash functions, symmetric encryption and
polynomial secret sharing are feasible on embedded platforms. The
\gls{GASE}~\cite{gase2023} relies on aggregated \glspl{MAC} and threshold
techniques to reduce computational overhead, while the \gls{LPKM}
protocol~\cite{lpkm2013} enables scalable and pairwise key establishment with
low storage requirements and distributed revocation. Symmetric encryption --
notably \gls{AES} -- has been proven to be practical on ESP32 devices without
prohibitive performance or energy costs, as demonstrated by the MCSC-WoT
framework~\cite{mcsc-wot}.\\

Other lines of work approach privacy-preservation by relying on hardware
fingerprints and pseudonym systems, including protocols based on
\glspl{PUF}~\cite{twofa2019,plgakd2021,healthcare2023}. However, these
approaches assume specialized hardware or centralized verification
infrastructure, greatly limiting their applicability to commodity devices and
infrastructure-less deployments. Similarly, zero-preloading key agreement
schemes~\cite{lightweight2008} attempt to eliminate prior trust establishment
but face unresolved challenges related to authentication bootstrapping and
revocation.\\

Several proposals focus instead on blockchain technology or distributed ledgers
to address \gls{iot} security concerns, including access
control~\cite{towards2019} and decentralized
\glspl{PKI}~\cite{decentralized2018pki}. While these approaches remove single
points of failure, they assume persistent connectivity incompatible with
disaster scenarios and computational capabilities far beyond those available in
low-end embedded platforms; pairing-based decentralized encryption schemes
further exacerbate this issue~\cite{privacy2012}.\\

Ultimately, existing work confirms the feasibility of lightweight security
mechanisms on constrained devices, but also highlights a persistent reliance on
infrastructure availability or pre-established trust. Rather than addressing
these challenges directly, the proposed \gls{ubabel} framework focuses on
providing a foundation for resilient and decentralized communication on top of
which security mechanisms may be integrated.
