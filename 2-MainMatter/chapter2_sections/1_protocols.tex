%!TEX root = ../../template.tex

\section{Wireless Communication Technologies}
\label{sec:wireless_comms}
\todo{FIXED: moved protocols para aqui + MQTT and CoAP overview}

A core pillar of resilient and decentralized \gls{iot} systems is the
heterogeneity of available communication capabilities and wireless protocols
across different layers of the communication stack.
Different technologies naturally expose different trade-offs in terms of
infrastructure requirements, range, throughput, discovery mechanisms, and
energy consumption -- all of which directly influence the design choices around
their usage.

There is no single communication protocol that fully addresses the requirements
of infrastructure-free and disaster-resilient operation. As such, it is
important to establish the capabilities and limitations of each one, as well as
their potential for harmonious cooperation within a unified system.

\subsection{Application-Layer Protocols}

At the application layer, two messaging protocols stand out given their
wide-spread adoption in \gls{iot} deployments:

\begin{description}
	\item[\textbf{\gls{MQTT}:}] A lightweight publish/subscribe messaging
	      protocol designed for constrained devices and unreliable, low-bandwidth
	      networks. It relies on a centralized \emph{broker} to mediate
	      communication between publishers and subscribers -- essential, a
	      server.

	\item[\textbf{\gls{CoAP}~\cite{coap2012,rfc7252}:}] A RESTful,
	      request/response protocol designed for constrained environments.
	      Operates over \gls{UDP} and assumes the presence of well-known
	      endpoints or proxy servers for resource discovery and access.
\end{description}

Both protocols inherently assume the availability of stable endpoints, and rely
on continuous network connectivity in order to operate. These assumptions
conflict with the infrastructure-less and failure-prone environments targeted
by \gls{ubabel}. For this reason, while such protocols are of interest for
broader interoperability (see Non-Functional Requirements in
Section~\ref{subsubsec:nonfunc_reqs}), they are not a part of the foundational
protocols employed by this work.

\subsection{Physical- and Link-Layer Technologies}

This section reviews the physical- and link-layer wireless technologies
relevant to resilient and decentralized \gls{iot} deployments. While the
previous section addressed higher-layer messaging protocols, the technologies
discussed here provide the foundational communication stack that determines the
behavior of \gls{ubabel}.

Table~\ref{tab:protocol_specs} details the basic operational characteristics of
the wireless communication technologies considered in this work and how they
drive protocol selection based on different factors:

\rowcolors{1}{}{GhostWhite}
\begin{table}[htbp]
	\centering
	\caption{Protocol Characteristics}
	\label{tab:protocol_specs}
	\footnotesize
	\begin{tabular}{@{}lccccc@{}}
		\toprule
		\rowcolor{Gainsboro}%
		\textbf{Protocol}  & \textbf{Discovery}        & \textbf{Range$^1$} & \textbf{Frequency}  & \textbf{Data Rate} & \textbf{Power Draw} \\
		\midrule
		Wi-Fi              & \glslink{SSID}{SSID} scan & 50-100 m           & 2.4 GHz             & 1-50 Mbps          & High                \\
		\glslink{BLE}{BLE} & Advertising               & 10-50 m            & 2.4 GHz             & 1-2 Mbps           & Very Low            \\
		\gls{ESP-NOW}      & Peer list                 & 50-100 m           & 2.4 GHz             & ~1 Mbps            & Low                 \\
		\gls{ZigBee}       & Beacon                    & 10-100 m           & 2.4 GHz             & 250 Kbps           & Very Low            \\
		\gls{LoRa}         & Preamble detection        & 2-10 km            & 433/868/915 MHz$^2$ & 0.3-50 Kbps$^3$    & Low$^3$             \\
		\bottomrule
		\multicolumn{6}{p{\linewidth}}{\footnotesize
			$^1$ Outdoor line-of-sight; indoor ranges typically 30-50\% lower
		}                                                                                                                                    \\
		\multicolumn{6}{p{\linewidth}}{\footnotesize
			$^2$ Frequency depends on regional regulations
		}                                                                                                                                    \\
		\multicolumn{6}{p{\linewidth}}{\footnotesize
			$^3$ Data rate and power consumption depend on \gls{SF}; higher \gls{SF} = longer range, lower rate, higher energy cost
		}                                                                                                                                    \\
	\end{tabular}
\end{table}
\todo{get actual milliamp power draw values from somewhere and/or source?}

\paragraph{Range vs. Throughput Trade-offs:} \gls{LoRa} provides 2-10 km range
(with appropriate antenna configurations) but very limited throughput (0.3-50
Kbps), suitable for sparse periodic synchronization between distant partitions.
\gls{BLE} and \gls{ESP-NOW} offer moderate range (50-100 m) with higher
throughput (1-2 Mbps), appropriate for dense local deployments where multi-hop
forwarding is feasible and often required.

\paragraph{Discovery Mechanisms:} Different protocols employ distinct peer
discovery approaches. \gls{BLE} advertising enables continuous passive
discovery, \gls{ESP-NOW} requires explicit peer lists, and \gls{ZigBee} uses
beacon-based association. \gls{ubabel}'s hybrid discovery mechanism
(Section~\ref{component:peer_discovery}) will accommodate these differences
while translating protocol-specific information into standardized peer
descriptors.

\paragraph{Energy Considerations:} Protocol selection must account for energy
costs: Wi-Fi consumes significantly more power than \gls{BLE}, while
\gls{LoRa} power consumption depends heavily on \gls{SF} configuration. The
data compression mechanisms (Section~\ref{component:reduction_compression})
interact with protocol selection to minimize overall energy expenditure.\\

Together, these protocol characteristics illustrate the inherent trade-offs
faced by resilient \gls{iot} deployments operating without reliable
infrastructure. Understanding how range, throughput, discovery mechanisms and
energy consumption vary across communication technologies is essential to
establish an architecture that can adaptively leverage multiple protocols
instead of relying on a single one.

\subsection{Discussion}
\label{subsec:rw_discussion_1}

The considered technologies highlight that no single protocol can
simultaneously optimize for range, throughput and energy efficiency upon loss
of infrastructure. Instead, resilience emerges from the ability to combine the
complimentary features of each technology and adapt communication strategies to
the specific context and requirements of a given environment.

These observations motivate an approach in which diversity is treated as a
first-class requirement, and provide the technological context for the
multi-protocol operation and architectural choices discussed in
Chapter~\ref{cha:solution_arch}.
