%!TEX root = ../../template.tex

\section{Security and Privacy for Resource-Constrained Emergency Communication}
\label{sec:security_privacy}

Disaster scenarios inherently involve sensitive data, such as location tracking
for search-and-rescue operations, environmental conditions for risk assessment,
among many others.

Unlike conventional \gls{iot} deployments where security infrastructure can be
carefully provisioned, emergency-focused deployments must balance security
guarantees against the possibility of infrastructure failure.

Traditional security approaches that assume persistent connectivity to
\glspl{CA}, \glspl{KDC}, or even blockchain networks cannot fully function, if
at all, when those very infrastructures collapse. Embedded platforms further
exacerbate this issue by ruling out computationally expensive cryptographic
approaches that could more easily provide a given network with a certain degree
of independence from those structures.

\subsection{Authentication and Privacy}
\label{subsec:authentication_privacy}

Authentication protocols are necessary to establish device identity and enable
secure communications. In disaster scenarios, authentication mechanisms must
operate without centralized coordinators while maintaining privacy guarantees
that prevent device tracking, activity correlation, and injection of false
information into the system by adversaries.
\todo{FIXED: added false info injection}

\subsubsection*{Lightweight Group Authentication}

The \gls{GASE}~\cite{gase2023} provides lightweight group authentication
suitable for resource-constrained devices through a combination of \gls{SSS}
and aggregated \gls{MAC} usage. It operates within a three-tier cloud-edge-IoT
architecture, focusing on asynchronous mass authentication of the low-end
\gls{iot} nodes therein.

There are several phases to this protocol, outlined as follows:
\begin{enumerate}
    \item \textbf{Initialization:} The \gls{AS} generates $2N$ secret-shadows,
        and distributes them to each \gls{iot} node (two per node). Nodes are
        then divided into $L$ groups, each with a unique $(t-1)$-degree
        polynomial along with a random value $r$ used to compute share tokens;
    \item \textbf{Hashed-shares reveal:} Within a given time window $w$,
        $(t-1)$ nodes reveal \emph{one} of their secret shares to the \gls{GL};
    \item \textbf{Group leader authentication:} The \gls{GL} uses Lagrange
        interpolation to recover the polynomial secret from the revealed
        shares, verifies authenticity, and all remaining participants derive
        sessions keys from the recovered secret;
    \item \textbf{Server authentication:} The \gls{GL} combines all node tags
        (\glspl{MAC} derived from session key) and sends them to the edge
        entity, which collects all \glspl{GL} tags and aggregates them into a
        single one via XOR operations, the \gls{AS} receives this
        \emph{Agg-MAC} and verifies the tag.
\end{enumerate}

The protocol's computational efficiency stems from requiring only hash
operations and modular arithmetic, avoiding expensive elliptic curve
operations.

For key updates the \gls{AS} randomly generates a new $r$, a new \gls{GL}, and
a new polynomial parameter for each group, which are published to allow nodes
to compute fresh shares from their stored secret without needing a secure
channel for updates or redistribution.

The main limitation of this approach is that it assumes persistent availability
of its centralized infrastructure components: the \gls{AS}, \glspl{GL} and edge
entities for aggregation. If the \gls{GL} becomes inactive, the authentication
process cannot proceed. While \gls{RPi} gateways could assume the role of an
\gls{AS} for initial provisioning, the \gls{GL} dependency persists among
sensor nodes themselves and isn't trivially solvable within the proposed
architecture.

Nevertheless, the individual mechanisms are still quite valuable: the
\emph{Agg-MAC} pattern efficiently aggregates multiple authentications, the
threshold approach ($t$-out-of-$n$ nodes must participate for group
authentication) provides fault tolerance against partial node failures, and the
session key derivation combining group secrets and device-specific keys
prevents node impersonation.

\subsubsection*{Privacy-Preserving Authentication}

\glspl{PUF} provide a distinctive hardware fingerprint to devices by taking
advantage of natural random variations present in integrated circuits, thus
allowing for the derivation of pseudonyms which are a common approach to
privacy in the context of \gls{iot} deployments.

Existing privacy-preserving authentication protocols based on \glspl{PUF} --
whether using challenge-response pairs~\cite{twofa2019,plgakd2021} or pseudonym
systems~\cite{healthcare2023} -- assume either specialized hardware unavailable
on ESP32/Raspberry Pico platforms  to leverage \glspl{PUF}, or centralized
verification infrastructure (pseudonym servers, credential authorities). Both
assumptions are incompatible with \gls{ubabel}'s target of commodity hardware
and infrastructure-less deployment model.


\subsection{Distributed Key Management}
\label{subsec:distrib_key_man}

The challenge of establishing shared cryptographic keys across
resource-constrained devices without central coordination has received
substantial attention, though most approaches make assumptions incompatible
with disaster scenarios.

\subsubsection*{Polynomial-Based Key Management}

The \gls{LPKM} protocol~\cite{lpkm2013} provides a compelling foundation for
embedded key distribution. Using polynomial evaluation on 8MHz ATmega128L
microcontrollers, this approach generates 128-bit group keys in 2-16
milliseconds with storage requirements of only 496-1616 bytes (depending on
security parameter $k$).

The proposed scheme supports multiple key types within a unified framework:
pairwise keys between (non-)neighboring nodes, cluster keys for local groups,
and group keys for larger networks. Storage complexity is O($k+1$) coefficients
per node, independent of network or group size, ensuring scalability.

No less critical for disaster scenarios -- where the possibility of malicious
agents must still be taken into account -- \gls{LPKM} enables distributed
revocation without central coordination. When a node is compromised, legitimate
nodes can independently compute updated keys that exclude the revoked member,
with no re-keying delay or coordinator involvement. Periodic share updating
provides backward secrecy through timer-based refresh operations that require
no coordination.

This approach does assume secure bootstrapping via a \gls{KDC} that preloads
polynomial shares into devices before deployment. For planned
disaster-resilience installations (e.g., campus sensor networks, building
monitoring systems), this physical preloading is acceptable, and a trusted
device (laptop, \gls{PDA}, etc.) can serve as a \gls{KDC}, keeping in line with
our zero-config goals.

\subsubsection*{Zero-Preloading Approaches}

Alternative schemes attempt to eliminate pre-distributed keys entirely. The
lightweight distributed key agreement protocol presented
in~\cite{lightweight2008} achieves a considerable speedup compared to
Diffie-Hellman by employing hash functions and bit-wise comparisons rather than
modular exponentiation.

The approach generates temporary key pairs on-demand through random secret
number generation, with nodes deriving shared keys via hash-based operations
and bit-wise comparisons of prefix bit-strings, thus eliminating the need for
pre-distributed keys.

However, it requires an inter-sensor authentication protocol to secure the
initial public key exchange, creating a circular dependency: authentication
requires keys, but key establishment requires authentication. Furthermore,
revocation mechanisms are only briefly mentioned without any concrete details,
limiting the potential of this approach.

This highlights a fundamental limitation in zero-config security: establishing
trust without pre-shared secrets or trusted third parties is theoretically
impossible\todo{too harsh? posso tar errado, never say never 2.0}. We accept
this limitation and opt for provisioning during deployment.

\subsection{Lightweight Encryption on Embedded Platforms}
\label{subsec:lightweight_enc}

The MCSC-WoT framework~\cite{mcsc-wot} demonstrates that \gls{AES} encryption
can operate efficiently on ESP32 platforms while maintaining multi-channel
communication. Their implementation measures encryption overhead on actual
hardware and  validates that symmetric cryptography remains viable on
resource-constrained devices without prohibitive energy or computational
costs.

This confirms that \gls{ubabel} can employ \gls{AES}-based encryption
for confidentiality without posing a significant compromise to battery lifetime
or real-time performance required for disaster scenarios. The coordination and
synchronization challenges associated with their master-based approach have
been discussed separately in Section~\ref{subsec:multi-channel}.

\subsection{Centralized Security Approaches (Incompatible with Disasters)}
\label{subsec:centr_sec}

Several recent approaches leverage blockchain or centralized infrastructure to
solve \gls{iot} security challenges, but their dependencies render them
unsuitable for disaster scenarios.

The blockchain-based access control system in~\cite{towards2019} moves Policy
Decision Points onto distributed ledgers, eliminating single points of failure
in access control. However, it assumes continuous connectivity to blockchain
nodes and cloud storage for off-chain data, failing precisely when
infrastructure collapses.

Similarly, the decentralized \gls{PKI} approach using blockchain-based
Name/Value Storage presented in~\cite{decentralized2018pki} replaces \glspl{CA}
with distributed blockchain nodes. While removing central \gls{CA}
dependencies, it still requires Internet connectivity for NVS queries and
assumes blockchain nodes remain reachable, which invalidates their usage during
disasters.

Decentralized \gls{ABE} schemes such as the one in~\cite{privacy2012} eliminate
central authorities through multi-authority cryptography but rely on bilinear
pairings with computational costs far exceeding the capabilities of our
targeted platforms.

\subsection{Discussion}
\label{subsec:rw_discussion_5}

Existing lightweight security mechanisms demonstrate feasibility on
resource-constrained platforms but assume infrastructure availability
incompatible with disaster scenarios.


The minimal viable set of mechanisms 
\todo{FIXED: identificar o minimal set of mechanisms}
to adopt comprises: (1) efficient authentication primitives using hash
operations and secret-sharing, (2) polynomial-based key management minimal
storage complexity, (3) distributed revocation without coordinator involvement,
(4) AES encryption validated on embedded platforms.

\gls{GASE}~\cite{gase2023} provides efficient group authentication using only
hash operations and modular arithmetic with valuable mechanisms (MAC
aggregation, threshold $(t-1)$-of-$n$ fault tolerance, session key derivation
preventing impersonation), but depends on persistent availability of
centralized components (\gls{AS}, gls{GL}, edge entities for aggregation).

\gls{LPKM}~\cite{lpkm2013} enables distributed key management with fast key
generation on 8MHz microcontrollers, O($k+1$) storage complexity independent of
network size, and distributed revocation without coordinator involvement, but
requires secure \gls{KDC} bootstrapping during pre-deployment, which is a
reasonable requirement.

Privacy-preserving authentication protocols based on
\glspl{PUF}~\cite{twofa2019,plgakd2021} and/or pseudonym
systems~\cite{healthcare2023} assume either specialized hardware (unavailable
on ESP32/Pico) or centralized verification infrastructure (pseudonym servers,
credential authorities).

Zero-preloading key agreement approaches~\cite{lightweight2008} face circular
dependencies (authentication requires keys, key establishment requires
authentication) without concrete revocation mechanisms.

AES encryption operates efficiently on ESP32 (validated by
MCSC-WoT~\cite{mcsc-wot}) without prohibitive energy costs.

Blockchain-based approaches (access control~\cite{towards2019}, decentralized
\gls{PKI}~\cite{decentralized2018pki}) eliminate single points of failure but
require continuous connectivity to distributed ledgers; decentralized ABE
schemes~\cite{privacy2012} rely on bilinear pairings exceeding embedded
platform capabilities.\\

No existing work integrates distributed authentication, key management, and
encryption for fully autonomous post-disaster operation while accepting
pre-disaster physical provisioning as a pragmatic bootstrapping mechanism.

\gls{ubabel}'s hybrid security model (Section
\ref{subsec:microbabel_approach_5}) distinguishes pre-disaster provisioning
(\gls{LPKM} shares, \gls{GASE} secret-shadows via \gls{KDC}) from post-disaster
autonomous operation (adapted threshold authentication with local \gls{GL}
election, gossip-based revocation, \gls{AES} encryption with
compress-then-encrypt).
