%!TEX root = ../../template.tex

\section{Summary}
\label{sec:rw_summary}

Table~\ref{tab:rw_summary} summarizes the key architectural differences between
existing work and \gls{ubabel}'s planned vision across the technical domains
discussed in this chapter.

\rowcolors{1}{}{GhostWhite}
\begin{table}[htbp]
\centering
\caption{Related work summary}
\label{tab:rw_summary}
\small
\begin{tabular}{@{}lp{5.5cm}p{5.5cm}@{}}
\toprule
\rowcolor{Gainsboro}%
\textbf{Domain} & \textbf{Related Work Limitations} & \gls{ubabel} Approach \\
\midrule
Protocol coordination & Master nodes (MCSC-WoT), SDN controllers (MINOS),
gateway coordinators (multi-protocol gateway) & Fully distributed selection
with hysteresis control \\
\midrule
Mesh topology & Tree-based restrictions, TCP-dependent links (HyParView),
centralized coordination & Connectionless BLE/LoRa discovery, hybrid views,
true P2P routing \\
\midrule
Time sync & Master beacons (MCSC-WoT), single-protocol gossip (RGCS) &
Protocol-specific Poisson rates, multi-graph converge-to-max \\
\midrule
Compression & Single-protocol optimization (Ambrosia, Sprintz), sender-focused
& Protocol-aware adaptive thresholds, gateway receiver energy management \\
% \midrule
% Authentication & Centralized components (GASE AS/GLs), specialized hardware
% (PUFs) & Two-phase: pre-disaster provisioning, post-disaster local GL election \\
% \midrule
% Key management & KDC bootstrapping (LPKM), circular dependencies
% (zero-preloading) & Pre-deployment shares distribution, gossip-based revocation \\
\bottomrule
\end{tabular}
\end{table}
\todo{FIXED: removed seurity row}

A common pattern is present across these domains: existing approaches rely on
centralized coordination mechanisms that become single points of failure during
infrastructure collapse.

Protocol coordination depends on master nodes or \gls{SDN} controllers, mesh
topologies assume consistent connectivity or gateway coordinators, time
synchronization requires master beacons, and authentication protocols depend on
centralized servers or group leaders.

Even distributed approaches like HyParView and \gls{RGCS} make assumptions
(routing infrastructure, persistent \gls{TCP} connections, single-protocol
homogeneity) incompatible with resource-constrained multi-protocol disaster
scenarios.

In the next Chapter we discuss how these limitations inform the design of
\gls{ubabel} as motivating factors for developing a fully decentralized
architecture -- one that prioritizes autonomous operation, protocol diversity,
and resilience under unstable connectivity.
\todo{FIXED: closing paragraph, mais narrativa etc?}

There we highlight the system requirements, operating conditions, and
architectural choices that will enable robust communication and coordination
across heterogeneous devices.

% \section{Summary of Related Work}
% \label{sec:summary_related_work3}
%
% \begin{table}[htbp]
% \centering
% \caption{Feature comparison: Related work vs. MicroBabel}
% \label{tab:feature_matrix}
% \footnotesize
% \begin{tabular}{@{}lccccccc@{}}
% \toprule
% \textbf{Feature} & \textbf{MCSC-WoT} & \textbf{MINOS} & \textbf{HyParView} & \textbf{RGCS} & \textbf{Ambrosia} & \textbf{GASE} & \gls{ubabel} \\
% \midrule
% Multi-protocol & \checkmark & \checkmark & \xmark & \xmark & \xmark & \xmark & \checkmark \\
% Masterless & \xmark & \xmark & \checkmark & \checkmark & n/a & \xmark & \checkmark \\
% Embedded platform & \checkmark & \xmark & \xmark & \checkmark & \checkmark & \checkmark & \checkmark \\
% P2P mesh & \xmark & \xmark & \checkmark & \checkmark & n/a & \xmark & \checkmark \\
% Adaptive compression & \xmark & \xmark & n/a & n/a & \checkmark & n/a & \checkmark \\
% Infrastructure-less & \xmark & \xmark & \checkmark & \checkmark & \xmark & \xmark & \checkmark \\
% Distributed auth & \xmark & \xmark & n/a & n/a & n/a & \texttildelow & \checkmark \\
% \bottomrule
% \multicolumn{8}{l}{\footnotesize \texttildelow = Partial support (requires centralized components)} \\
% \multicolumn{8}{l}{\footnotesize n/a = Not applicable to this work's scope}
% \end{tabular}
% \end{table}
%

% \bgroup
% \rowcolors{1}{}{GhostWhite}
% \begin{xltabular}{\textwidth}{Xccccc}
%   \caption{Test results summary.}
%   \label{tab:hla:results}\\
%   \toprule
%   \rowcolor{Gainsboro}%
%   Test   & Anomalies  & Warnings  & Correct   & Categories            & Missed \\
%   \midrule
% Connection~\cite{Beckman08}     & 2       & 2          & 1          & \emph{C}              & 1 \\
% Coordinates'03~\cite{Artho03}   & 1        & 4          & 1          & \emph{2B, 1C}          & 0 \\
% Local Variable~\cite{Artho03}    & 1        & 2          & 1          & \emph{A}              & 0 \\
% NASA~\cite{Artho03}              & 1        & 1          & 1          & ---                    & 0 \\
% Coordinates'04~\cite{Artho04}    & 1        & 4          & 1          & \emph{3C}              & 0 \\
% Buffer~\cite{Artho04}            & 0        & 7          & 0          & \emph{2A, 1B, 2C, 2D}  & 0 \\
% Double-Check~\cite{Artho04}      & 0        & 2          & 0          & \emph{1A, 1B}          & 0 \\
% StringBuffer~\cite{Flanagan04}  & 1        & 0          & 0          & ---                    & 1 \\
% Account~\cite{Praun03}          & 1        & 1          & 1          & ---                   & 0 \\
% Jigsaw~\cite{Praun03}            & 1        & 2          & 1          & \emph{C}              & 0 \\
% Over-reporting~\cite{Praun03}    & 0        & 2          & 0          & \emph{1A, 1C}          & 0 \\
% Under-reporting~\cite{Praun03}  & 1        & 1          & 1          & ---                    & 0 \\
% Allocate Vector~\cite{IBM-Rep}  & 1        & 2          & 1          & \emph{C}              & 0 \\
% Knight Moves~\cite{Beckman08}   & 1        & 3          & 1          & \emph{2B}              & 0 \\
%   \midrule
%   \rowcolor{Gainsboro}%
% Total                            & 12      & 33        & 10        & 5A, 6B, 10C, 2D       & 2 \\
%   \bottomrule
%   \end{xltabular}
% \egroup
