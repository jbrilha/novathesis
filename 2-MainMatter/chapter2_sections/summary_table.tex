%!TEX root = ../../template.tex

\section{Summary}
\label{sec:rw_summary}

Table~\ref{tab:rw_summary} summarizes the key architectural differences between
existing work and \gls{ubabel}'s planned vision across the technical domains
discussed in this chapter.

\rowcolors{1}{}{GhostWhite}
\begin{table}[htbp]
	\centering
	\caption{Related work summary}
	\label{tab:rw_summary}
	\small
	\begin{tabular}{@{}lp{5.5cm}p{5.5cm}@{}}
		\toprule
		\rowcolor{Gainsboro}%
		\textbf{Domain}                               & \textbf{Related Work Limitations}                                    & \gls{ubabel} Approach \\
		\midrule
		Radio coordination                            & Master nodes (MCSC-WoT), \gls{SDN} controllers (MINOS),
		gateway coordinators (multi-protocol gateway) & Fully distributed selection, adaptation based on application context                         \\
		\midrule
		Mesh topology                                 & Tree-based restrictions, \gls{TCP}-dependent links (HyParView),
		centralized coordination                      & Connectionless \gls{BLE}/\gls{LoRa} discovery, hybrid views,
		true P2P routing                                                                                                                             \\
		\midrule
		Time sync                                     & Master beacons (MCSC-WoT), single-protocol gossip (\gls{RGCS})       &
		Protocol-specific Poisson rates, converge-to-max                                                                                 \\ 
		% \midrule
		% Compression & Single-protocol optimization (Ambrosia, Sprintz), sender-focused
		% & Protocol-aware adaptive thresholds, gateway receiver energy management \\
		% \midrule
		% Authentication & Centralized components (GASE AS/GLs), specialized hardware
		% (PUFs) & Two-phase: pre-disaster provisioning, post-disaster local GL election \\
		% \midrule
		% Key management & KDC bootstrapping (LPKM), circular dependencies
		% (zero-preloading) & Pre-deployment shares distribution, gossip-based revocation \\
		\bottomrule
	\end{tabular}
\end{table}

A common pattern is present across these domains: existing approaches rely on
centralized coordination mechanisms that become single points of failure during
infrastructure collapse.

Protocol coordination depends on master nodes or \gls{SDN} controllers, mesh
topologies assume consistent connectivity or gateway coordinators, time
synchronization requires master beacons.

Even distributed approaches like HyParView and \gls{RGCS} make assumptions
(routing infrastructure, persistent \gls{TCP} connections, single-protocol
homogeneity) incompatible with resource-constrained multi-protocol disaster
scenarios.

In the next Chapter we discuss how these limitations inform the design of
\gls{ubabel} as motivating factors for developing a fully decentralized
architecture -- one that prioritizes autonomous operation, protocol diversity,
and resilience under unstable connectivity.

There we highlight the system requirements, operating conditions, and
architectural choices that will enable robust communication and coordination
across heterogeneous devices.
