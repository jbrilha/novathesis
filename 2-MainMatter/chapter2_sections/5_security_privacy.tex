%!TEX root = ../../template.tex

\section{Security and Privacy for Resource-Constrained Emergency Communication}
\label{sec:security_privacy}

Disaster scenarios inherently involve sensitive data, such as location tracking
for search-and-rescue operations, environmental conditions for risk assessment,
among many others.

Unlike conventional \gls{iot} deployments where security infrastructure can be
carefully provisioned, emergency-focused deployments must balance security
guarantees against the possibility of infrastructure failure.

Traditional security approaches that assume persistent connectivity to
\glspl{CA}, \glspl{KDC}, or even
blockchain networks cannot fully function, if at all, when those very
structures collapse. Embedded platforms further exacerbate this issue by ruling
out computationally expensive cryptographic approaches that could more easily
provide a given network with a certain degree of independence from those
structures.

\subsection{Authentication and Privacy}
\label{subsec:authentication_privacy}

Authentication protocols are necessary to establish device identity and enable
secure communications. In disaster scenarios, authentication mechanisms must
operate without centralized coordinators while maintaining privacy guarantees
that prevent device tracking and activity correlation.

\subsubsection*{Lightweight Group Authentication}

The \gls{GASE}~\cite{gase2023} demonstrates lightweight group authentication
suitable for resource-constrained devices through a combination of \gls{SSS}
and aggregated \gls{MAC} usage. It operates within a three-tier cloud-edge-IoT
architecture, focusing on asynchronous mass authentication of the low-end
\gls{iot} nodes therein.

There are several phases to this protocol, summarily:
\begin{enumerate}
    \item \textbf{Initialization:} The \gls{AS} generates $2N$ secret-shadows,
        and distributes them to each \gls{iot} node (two per node). Nodes are
        then divided into $L$ groups, each with a unique $(t-1)$-degree
        polynomial along with a random value $r$ used to compute share tokens;
    \item \textbf{Hashed-shares reveal:} Within a given time window $w$,
        $(t-1)$ nodes reveal \emph{one} of their secret shares to the \gls{GL};
    \item \textbf{Group leader authentication:} The \gls{GL} uses Lagrange
        interpolation to recover the polynomial secret from the revealed
        shares, verifies authenticity, and all remaining participants derive
        sessions keys from the recovered secret;
    \item \textbf{Server authentication:} The \gls{GL} combines all node tags
        (\glspl{MAC} derived from session key) and sends them to the edge
        entity, which collects all \glspl{GL} tags and aggregates them into a
        single one via XOR operations, the \gls{AS} receives this
        \emph{Agg-MAC} and verifies the tag.
\end{enumerate}

The protocol's computational efficiency stems from requiring only hash
operations and modular arithmetic, avoiding expensive elliptic curve
operations.

For key updates the \gls{AS} randomly generates a new $r$, a new \gls{GL}, and
a new polynomial parameter for each group, which are published to allow nodes
to compute fresh shares from their stored secret without needing a secure
channel for updates or redistribution.

The main issue with this approach is that it assumes persistent availability of
its centralized infrastructure components: the \gls{AS}, \glspl{GL} and edge
entities for aggregation. If the \gls{GL} becomes inactive, the authentication
process cannot proceed. While \gls{RPi} gateways could assume the role of an
\gls{AS} for initial provisioning, the \gls{GL} dependency persists among
sensor nodes themselves and isn't trivially solvable within the proposed
architecture.

Nevertheless, the individual mechanisms are still quite valuable: the
\emph{Agg-MAC} pattern efficiently aggregates multiple authentications, the
threshold approach ($t$-out-of-$n$ nodes must participate for group
authentication) provides fault tolerance against partial node failures, and the
session key derivation combining group secrets and device-specific keys
prevents node impersonation.

\subsubsection*{Privacy-Preserving Authentication}

\glspl{PUF} provide a distinctive hardware fingerprint to devices by taking
advantage of natural random variations present in integrated circuits, thus
allowing for the derivation of pseudonyms which are a common approach to
privacy in the context of \gls{iot} deployments.

Existing privacy-preserving authentication protocols based on \glspl{PUF} --
whether using challenge-response pairs~\cite{twofa2019,plgakd2021} or pseudonym
systems~\cite{healthcare2023} -- assume either specialized hardware unavailable
on ESP32/Raspberry Pico platforms  to leverage \glspl{PUF}, or centralized
verification infrastructure (pseudonym servers, credential authorities). Both
assumptions are incompatible with \textbf{\gls{ubabel}}'s commodity hardware
and infrastructure-less deployment model.


\subsection{Distributed Key Management}
\label{subsec:distrib_key_man}

The challenge of establishing shared cryptographic keys across resource-
constrained devices without central coordination has received substantial
attention, though most approaches make assumptions incompatible with disaster
scenarios.

\subsubsection*{Polynomial-Based Key Management}

The \gls{LPKM} protocol~\cite{lpkm2013}
provides a compelling foundation for embedded key distribution. Using
polynomial evaluation on 8MHz ATmega128L microcontrollers, this approach
generates 128-bit group keys in 2-16 milliseconds with storage requirements of
only 496-1616 bytes (depending on security parameter $k$).

The proposed scheme supports multiple key types within a unified framework:
pairwise keys between (non-)neighboring nodes, cluster keys for local groups,
and group keys for larger networks. Storage complexity is O($k+1$) coefficients
per node, independent of network or group size, ensuring scalability.

No less critical for disaster scenarios -- where the possibility of malicious
agents must still be taken into account -- \gls{LPKM} enables distributed revocation
without central coordination. When a node is compromised, legitimate nodes can
independently compute updated keys that exclude the revoked member, with no
re-keying delay or coordinator involvement. Periodic share updating provides
backward secrecy through timer-based refresh operations that require no
coordination.

This approach does assume secure bootstrapping via a \gls{KDC} that preloads
polynomial shares into devices before deployment. For planned
disaster-resilience installations (campus sensor networks, building monitoring
systems), this physical preloading is acceptable, and a trusted device (laptop,
\gls{PDA}, etc.) can serve as a \gls{KDC}, keeping in line with our zero-config
efforts.

\subsubsection*{Zero-Preloading Approaches}

Alternative schemes attempt to eliminate pre-distributed keys entirely. The
lightweight distributed key agreement protocol presented in
~\cite{lightweight2008} achieves a considerable speedup compared to
Diffie-Hellman by employing hash functions and bit-wise comparisons rather
than modular exponentiation.

The approach generates temporal key pairs on-demand through random secret
number generation, with nodes deriving shared keys via hash-based operations
and bit-wise comparisons of prefix bit-strings, thus eliminating the need for
pre-distributed keys.

However, it requires an inter-sensor authentication protocol to secure the
initial public key exchange, creating a circular dependency: authentication
requires keys, but key establishment requires authentication. Furthermore,
revocation mechanisms are only briefly mentioned without any concrete details,
limiting the potential of this implementation.

This highlights a fundamental limitation in zero-config security: establishing
trust without pre-shared secrets or trusted third parties is theoretically
impossible\todo{too harsh? posso tar errado}. We accept this limitation and opt
for provisioning during deployment.

\subsection{Lightweight Encryption on Embedded Platforms}
\label{subsec:lightweight_enc}

The MCSC-WoT framework~\cite{mcsc-wot} demonstrates that \gls{AES} encryption
can operate efficiently on ESP32 platforms while maintaining multi-channel
communication. Their implementation measures encryption overhead on actual
hardware and  validates that symmetric cryptography remains viable on
resource-constrained devices without prohibitive energy or computational
costs.

This confirms that \textbf{\gls{ubabel}} can employ \gls{AES}-based encryption
for confidentiality without posing a significant compromise to battery lifetime
or real-time performance required for disaster scenarios. The coordination
and synchronization challenges associated with their master-based approach have been discussed separately in Section~\ref{subsec:multi-channel}.

\subsection{Centralized Security Approaches (Incompatible with Disasters)}
\label{subsec:centr_sec}

Several recent approaches leverage blockchain or centralized infrastructure to
solve \gls{iot} security challenges, but their dependencies render them
unsuitable for disaster scenarios.

The blockchain-based access control system in~\cite{towards2019} moves Policy
Decision Points onto distributed ledgers, eliminating single points of failure
in access control. However, it assumes continuous connectivity to blockchain
nodes and cloud storage for off-chain data, failing precisely when
infrastructure collapses.

Similarly, the decentralized \gls{PKI} approach using blockchain-based
Name/Value Storage presented in~\cite{decentralized2018pki} replaces \glspl{CA}
with distributed blockchain nodes. While removing central \gls{CA}
dependencies, it still requires Internet connectivity for NVS queries and
assumes blockchain nodes remain reachable, which invalidates their usage during
disasters.

Decentralized \gls{ABE} schemes such as the one in~\cite{privacy2012} eliminate
central authorities through multi-authority cryptography but rely on bilinear
pairings with computational costs far exceeding the capabilities of our
targeted platforms.

\subsection{Implications for \textbf{\gls{ubabel}}}
\label{subsec:microbabel_implications_5}

Existing lightweight security mechanisms demonstrate feasibility on
resource-constrained platforms but assume infrastructure availability
incompatible with disaster scenarios.

\gls{GASE}~\cite{gase2023} provides efficient group authentication using only
hash operations and modular arithmetic with valuable mechanisms (MAC
aggregation, threshold $(t-1)$-of-$n$ fault tolerance, session key derivation
preventing impersonation), but depends on persistent availability of
centralized components (\gls{AS}, gls{GL}, edge entities for aggregation).

\gls{LPKM}~\cite{lpkm2013} enables distributed key management with fast key
generation on 8MHz microcontrollers, O($k+1$) storage complexity independent of
network size, and distributed revocation without coordinator involvement, but
requires secure \gls{KDC} bootstrapping during pre-deployment, which is a
reasonable requirement.

Privacy-preserving authentication protocols based on
\glspl{PUF}~\cite{twofa2019,plgakd2021} and/or pseudonym
systems~\cite{healthcare2023} assume either specialized hardware (unavailable
on ESP32/Pico) or centralized verification infrastructure (pseudonym servers,
credential authorities).

Zero-preloading key agreement approaches~\cite{lightweight2008} face circular
dependencies (authentication requires keys, key establishment requires
authentication) without concrete revocation mechanisms.

AES encryption operates efficiently on ESP32 (validated by
MCSC-WoT~\cite{mcsc-wot}) without prohibitive energy costs.

Blockchain-based approaches (access control~\cite{towards2019}, decentralized
\gls{PKI}~\cite{decentralized2018pki}) eliminate single points of failure but
require continuous connectivity to distributed ledgers; decentralized ABE
schemes~\cite{privacy2012} rely on bilinear pairings exceeding embedded
platform capabilities.\\

No existing work integrates distributed authentication, key management, and
encryption for fully autonomous post-disaster operation while accepting
pre-disaster physical provisioning as a pragmatic bootstrapping mechanism.

\textbf{\gls{ubabel}}'s hybrid security model (Section
\ref{subsec:microbabel_approach_5}) distinguishes pre-disaster provisioning
(\gls{LPKM} shares, \gls{GASE} secret-shadows via \gls{KDC}) from post-disaster
autonomous operation (adapted threshold authentication with local \gls{GL}
election, gossip-based revocation, \gls{AES} encryption with
compress-then-encrypt).

\subsection{MicroBabel's Approach}
\label{subsec:microbabel_approach_5}

While zero-config approaches face circular authentication dependencies and
centralized schemes fail when infrastructure collapses, \textbf{\gls{ubabel}}
will adopt a hybrid security model that distinguishes between pre-disaster
deployment and post-disaster autonomous operation.

\begin{description}
    \item \textbf{Pre-disaster deployment phase:} For planned installations
        (building sensor networks, campus-wide monitoring), devices are
        provisioned during initial deployment:

        \begin{itemize}
            \item \textbf{LPKM polynomial shares:} Preloaded via \gls{KDC} for
                distributed key management
            \item \textbf{GASE secret-shadows:} Two per node for threshold
                authentication ($(t-1)$-of-$n$ reveal protocol)
        \end{itemize}

        These are reasonable assumptions for disaster-resilience scenarios, as
        preparation happens before any emergency operation needs to take place.

	\item \textbf{Post-disaster autonomous operation:} Once a disaster occurs
	      and infrastructure fails, the network operates autonomously using
	      mechanisms that require no central coordination:

	      \begin{itemize}
              \item \textbf{Threshold-based authentication} (adapted from
                  \gls{GASE}): Nodes form ad-hoc authentication groups where
                  $(t-1)$-of-$n$ members reveal secret shares (pre-distributed)
                  to derive session keys via Lagrange interpolation. Local
                  coordinator election replaces GASE's centralized \glspl{GL}
                  Session key derivation combines group secrets with
                  device-specific keys to prevent impersonation, while
                  aggregated \gls{MAC} tags enable efficient multi-node
                  verification without individual authentication overhead.
                  \todo{too dense?}
              \item \textbf{Distributed revocation} (adapted from \gls{LPKM}):
                  nodes independently compute updated keys excluding
                  compromised peers, with revocation decisions propagated via
                  gossip
              \item \textbf{Periodic share updating} (adapted from \gls{LPKM}):
                  timer-based key refresh for backward secrecy, no coordinator
                  needed
              \item \textbf{Lightweight encryption}: \gls{AES} encryption
                  (demonstrated by MCSC-WoT on ESP32) with
                  compress-then-encrypt strategies
                  (Section~\ref{sec:data_compression}) to minimize ciphertext
                  size and transmission energy
              \item \textbf{Channel hopping/protocol switching for security}:
                  multi-channel/multi-protocol operation complicates
                  eavesdropping and uses masterless synchronization
                  (\gls{RGCS}) rather than master-based beacons
	      \end{itemize}

	\item \textbf{Threat model:} The system will prioritize availability and
	      resilience over complete compromise resistance. Assumptions include:

	      \begin{itemize}
		      \item \textbf{Honest majority}: most nodes behave correctly;
		            adversaries cannot compromise majority simultaneously
		      \item \textbf{Local adversary}: attackers can monitor/disrupt local
		            regions but not entire network simultaneously
		      \item \textbf{Physical security during deployment}: devices can be
		            provisioned securely before disaster strikes
	      \end{itemize}

	      This threat model reflects emergency priorities where maintaining
	      communication capability matters more than preventing all possible
	      attacks. If an adversary with sophisticated capabilities attacks during
	      a disaster, communication infrastructure would already be their target
	      regardless of security mechanisms.
\end{description}

