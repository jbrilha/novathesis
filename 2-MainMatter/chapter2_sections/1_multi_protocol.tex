%!TEX root = ../../template.tex

\section{Multi-Protocol Communication and Adaptive Protocol Selection}
\label{sec:multi_protocol}

Single-protocol (i.e. Wi-Fi only, \gls{BLE} only, \gls{LoRa} only, etc.) communication
architectures are susceptible to a fundamental vulnerability: if their chosen
medium becomes unavailable or inefficient due to interference, range
limitations, or infrastructure failures, the entire system loses connectivity
and most times ceases to function altogether.

This becomes especially grave in disaster scenarios where communication
conditions change unpredictably due to factors such as \gls{RF} interference
from debris, obstacles blocking line-of-sight propagation, or damage to access
points.

Existing approaches to multi-protocol \gls{iot} systems can be placed in three
categories: multi-channel resilience architectures that orchestrate
communication technologies for emergency scenarios; implementations that show
the feasibility of protocol heterogeneity on embedded platforms; and adaptive
selection frameworks that switch protocols based on runtime conditions.

\subsection{Multi-Channel Resilience Architectures}
\label{subsec:multi-channel}

Several disaster-focused systems explicitly address infrastructure failures
through protocol diversity.

The AWCT (Always Connected Things) framework~\cite{awct} orchestrates \gls{LPWAN}
on top of \gls{LoRaWAN} with ad-hoc networks (Bluetooth and Wi-Fi) specifically for
standby emergency communication. Their architecture adds three modules to
standard \gls{iot} devices (Raspberry Pi boards, in their test case): a battery module
for power management, a power interrupt handler that triggers emergency mode
when the power grid fails, and an ad-hoc bridge that forwards packets between
the Bluetooth, Wi-Fi and \gls{LPWAN} interfaces.

The system leverages dense \gls{iot} device deployment to provide emergency coverage,
demonstrating that existing infrastructure can still serve a purpose during
scenarios where the main power grid suffers issues.

However, AWCT's reliance on centralized \gls{LoRaWAN} gateways for Internet
connectivity creates a single point of failure when those gateways become
unreachable.\\

A more comprehensive heterogeneous approach is presented
in~\cite{heterogeneous-disaster}, which integrates \gls{HF} radio with
\gls{NVIS}, satellite links, \glspl{WSN}, and \gls{DTN} with mobile drones for
disaster monitoring. Their system uses \gls{RPL}~\cite{rfc6550} with three
separate instances to differentiate traffic by priority: human data (voice/text
via Bluetooth) receives the highest priority, followed by drone-collected data
and finally sensor data.

The \gls{NVIS} backhaul provides 250 km coverage radius without line-of-sight
requirements, offering a cost-effective alternative to satellite
communications.

While demonstrating successful real-world validation in Antarctica and urban
deployments, the architecture's core \gls{NVIS} topology with centralized coordination
contrasts fully distributed operational requirements. Furthermore, their \gls{WSN}
layer requires a minimum 20-second sending interval to maintain acceptable
packet loss rates at 10 hops, which highlights throughput limitations of
single-channel tree topologies.\\

Security-focused multi-channel approaches like MCSC-WoT~\cite{mcsc-wot} combine
\gls{AES} encryption with dynamic channel hopping across 2.4 GHz Wi-Fi channels
to defend against jamming and eavesdropping attacks. Their lightweight
synchronization mechanism minimizes energy consumption while maintaining
security through \gls{FHSS} patterns generated via \gls{PRNG}. Nodes that lose
synchronization can rejoin by hopping to random channels and waiting for the
next synchronization signal.

However, the system depends on a master node broadcasting synchronization
signals and uses an initial \gls{PRNG} seed shared among all nodes, raising
questions about scalability and seed distribution mechanisms -- particularly
how new nodes can acquire seeds if deployed to a network at a later time.

\subsection{Protocol Heterogeneity on Embedded Platforms}
\label{subsec:proto-het}

The work presented in~\cite{multiproto_gateway} shows the technical feasibility
of multi-protocol operation on commodity Wi-Fi/\gls{BLE} modules by integrating
ESP-NOW, ZigBee, and Modbus protocols on devices from the ESP32 family to
construct multi-hop, tree-based wireless networks.

Their implementation uses \gls{BLE} advertising beacons for neighbor discovery with
\gls{RSSI} measurements (for distance estimation), computing parent selection
priorities based on weighted combinations of child count, \gls{RSSI} values, and hop
count in the network tree of a given node. The system also supports automatic
parent reselection when the current parent becomes unreachable, making it
somewhat resilient to single node failures.

Testing demonstrated successful multi-hop operation up to 5 hops in office
environments, but evaluation was limited to linear network topologies.
Additionally, this architecture relies on a tree topology with master-slave
communication that is centered around a gateway (coordinator) rather than being
a \gls{P2P} mesh, creating a single point of failure at the coordinator
node.\\

\subsection{Runtime Adaptive Protocol Selection}
\label{subsec:runtime-adaptive}

\todo{este é o paper do MDPI, removo?}
A recent approach to protocol adaptation is presented in~\cite{adaptive2025},
which introduces a closed-loop control system with four integrated components:
a context monitor sampling runtime metrics at 1 Hz, a decision engine using
multi-criteria weighted scoring across six dimensions (message frequency,
payload size, network conditions, packet loss rate, energy budget, \gls{QoS}
requirements), protocol adapters encapsulating protocol-specific libraries, and
a learning component that adjusts thresholds using \gls{EWMA} for
deployment-specific patterns.

Some key aspects of this work include \gls{hysteresis} control with a threshold
band to prevent oscillation between protocols, and switching cost awareness
requiring benefits to exceed costs by a certain threshold before transitioning.

However, this framework assumes infrastructure availability for protocol
endpoints (\gls{MQTT} brokers, \gls{HTTP} servers, \gls{CoAP} endpoints) and was
evaluated only on laptop platforms using Python libraries rather than embedded
C implementations.

Nevertheless, the evaluation methodology provides valuable insight into the
energy expenditure of these different approaches (further explored in
Section~\ref{subsec:receiver_energy}), and grants useful foundations for more
complex multi-protocol approaches, in particular with the multi-criteria
decision framework and \gls{hysteresis} control mechanism, but the actual
system cannot operate when infrastructure fails.\\

The MINOS platform~\cite{minos2019} exemplifies the limitations of centralized
multi-protocol approaches. While providing sophisticated multi-protocol support
(CORAL-SDN and Adaptable-RPL) with dynamic protocol deployment and real-time
parameter tuning, the system depends fundamentally on a centralized \gls{SDN}
controller, \gls{MQTT} broker, and web server infrastructure. The architecture's
single point of failure means that when the controller becomes unreachable the
entire system loses its adaptive capabilities and reverts to static operation
at best, or complete failure at worst.

\subsection{Implications for \textbf{\gls{ubabel}}}
\label{subsec:microbabel_implications_1}

Existing frameworks demonstrate the technical feasibility of heterogeneous
channel/protocol coordination but rely on centralized control mechanisms
incompatible with disaster scenarios. AWCT~\cite{awct} depends on centralized
\gls{LoRaWAN} gateways for Internet connectivity, the heterogeneous disaster
architecture~\cite{heterogeneous-disaster} uses centralized \gls{NVIS} coordination
with minimum 20-second send intervals at 10 hops, MCSC-WoT~\cite{mcsc-wot}
requires master nodes broadcasting synchronization signals, the multi-protocol
gateway~\cite{multiproto_gateway} uses tree topologies with gateway
coordinators as single points of failure, and MINOS~\cite{minos2019} assumes
SDN controllers with persistent MQTT/web infrastructure.

While adaptive protocol selection frameworks~\cite{adaptive2025} provide
runtime switching strategies, they assume infrastructure availability
(\gls{MQTT} brokers, \gls{HTTP} servers, \gls{CoAP} endpoints) and have not
been validated on embedded platforms.\\

\todo{se calhar "no existing work" é too much? posso não ter encontrado um mega óbvio}
No existing work addresses fully distributed multi-protocol coordination where
nodes can autonomously select and switch protocols without persistent
infrastructure. \textbf{\gls{ubabel}}'s approach to decentralized protocol
selection is detailed in Section \ref{subsec:microbabel_approach_1}, with
masterless synchronization mechanisms addressed in
\ref{sec:decentralized_sync}.

\inlinetodo{acho que faz mais sentido passar todas as "microbabel approaches" para 
o próximo capítulo em vez de as ter aqui, então adicionei o parágrafo anterior para
esclarecer quais são as gaps dos related works em cada área --- yay or nay?
de qualquer forma deixo aqui as approaches caso contenham disparates imediatamente corrigíveis}

\subsection{MicroBabel's Approach}
\label{subsec:microbabel_approach_1}

While existing multi-channel approaches rely on centralized decision-making for
channel selection -- whether through master nodes (MCSC-WoT), \gls{SDN} controllers
(MINOS), or infrastructure endpoints -- \textbf{\gls{ubabel}} will address these
limitations through decentralized \emph{protocol} selection, extending beyond
single-medium channel hopping.

The system will employ opportunistic multi-protocol operation (\gls{BLE} + Wi-Fi +
\gls{LoRa} + ESP-NOW) wherein devices adaptively select communication mediums based
on factors like message priority, neighbor availability, energy budget, network
conditions and device capabilities. This selection will be performed through
local decision-making informed by information gathered through gossip
protocols, without requiring \gls{SDN} controllers or infrastructure coordination.

\Gls{hysteresis} control mechanisms adapted from the Adaptive Protocol Selection
Framework prevent oscillations in these decisions while allowing rapid
response to changing conditions.

Protocol-specific compression strategies account for the different
energy/bandwidth trade-offs across the proposed stack (e.g. aggressive
compression for (relatively) energy-expensive \gls{LoRa}, lighter compression for
short-range \gls{BLE}), integrating the receiver energy asymmetry observations into
power management decisions.

Unlike AWCT's \gls{LoRaWAN} gateway dependency, Heterogeneous \gls{iot}'s \gls{NVIS} centralized
backhaul, MCSC-WoT's master-based synchronization, or MINOS's \gls{SDN} controller
requirement, \textbf{\gls{ubabel}} operates autonomously in a \gls{P2P}
manner with no static coordinator dependencies.

The multi-protocol capability provides resilience through diversity rather than
optimization through centralized selection: when conditions render one medium
unsuitable, devices autonomously transition to alternatives without requiring
coordinator intervention.
