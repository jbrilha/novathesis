%!TEX root = ../../template.tex

\section{\gls{P2P} Mesh Networking and Topology Management}
\label{sec:p2p_mesh}

Infrastructure-dependent star topologies in which devices communicate via a
central coordinator or gateway suffer from a similar downside when compared
with single-protocol communication: when that coordinator becomes unavailable
or unreachable, the entire network loses connectivity.

This pattern pervades current \gls{iot} deployments, from Wi-Fi access point
dependencies to \gls{LoRaWAN} gateway requirements, and becomes catastrophic in
disaster scenarios where central coordinators are most likely to fail first,
\todo{FIXED: justify this claim}
due to power outages, physical damage, or severe network congestion from increased
emergency communications traffic.

\gls{P2P} mesh architectures address this limitation by distributing
coordination across all participating nodes, thus eliminating single points of
failure. Nonetheless, achieving robust mesh operation requires solving three
interconnected challenges: neighbor discovery and establishment of initial
connectivity, topology maintenance as nodes join/leave a network, and efficient
data routing through multi-hop paths when direct communication becomes
impossible or inefficient.

\subsection{Peer Discovery and Topology Maintenance}
\label{subsec:peer_disc}

As discussed in Section~\ref{subsec:proto-het}, the Multi-Protocol \gls{iot}
Gateway implementation~\cite{multiproto_gateway} demonstrates \gls{BLE}-based
neighbor discovery with \gls{RSSI} measurements for proximity estimation.

While that work focuses on multi-protocol integration, its topology management
reveals a limitation of its tree-based approach: the system implements
automatic parent reselection when coordinators fail, but the underlying tree
structure imposes that nodes can only communicate through their parent-child
relationships rather than arbitrary peer connections.

This restriction limits route diversity and resilience, creating dependency
chains where a single intermediate node failure can disconnect entire subtrees.\\

A more sophisticated approach to membership management is presented in
HyParView~\cite{hyparview2007}, in which each node maintains two distinct
partial views for scalability: a small active view (size = fanout + 1)
containing nodes with which symmetric links are actively maintained, and a
larger passive view serving as a backup pool of potential neighbors that may be
promoted to the active view if one of its nodes fails.

The active view is managed reactively, such that nodes are added during join
operations and removed upon failure detection, while the passive view is
maintained cyclically through periodic shuffle operations that randomly
exchange node identifiers between peers.

This hybrid strategy enables remarkable resilience, with the system recovering
from 80\% node failures with minimal reliability loss (maintaining ~95\%
reliability) and from 50\% failures in just 1-2 membership cycles, compared to
60+ cycles required by purely cyclic protocols like Cyclon~\cite{cyclon2005}.

HyParView's deterministic flooding approach, by broadcasting messages along the
entire active view graph rather than probabilistic neighbor selection, enables
fast failure detection since every active link is tested at each broadcast. The
symmetric link requirement ensures bidirectionality: if node A can reach node
B, then B can reach A, preventing the formation of one-way communication paths
that complicate routing.

However, HyParView assumes \gls{TCP} availability for maintaining persistent
connections and using connection failures as implicit failure detectors.
\todo{FIXED: adicionei isto com base na nota. mas nós vamos ter some form of multi-hop também so not sure I understand pqq é mau}
Additionally, it assumes that any node can communicate with any other node
in its active view, which in practice requires some form of routing infrastructure.

This dependency on full network stack functionality makes direct application to
resource-constrained embedded platforms challenging, though the architectural
principles of hybrid views and shuffle-based passive view maintenance remain
valuable.\\

The heterogeneous disaster \gls{iot} architecture~\cite{heterogeneous-disaster}
discussed in Section~\ref{subsec:multi-channel} employs \gls{RPL} for its
\gls{WSN} layer, demonstrating practical routing in resource-constrained
disaster scenarios.

Their performance analysis reveals considerable trade-offs: convergence time
scales linearly from 7 seconds for 20 nodes to 14.5 seconds for 100 nodes,
while \gls{PLR} at 10 hops reaches 80\% with a 10-second sending
interval but becomes acceptable at 20-second intervals.

These measurements highlight the throughput limitations imposed by tree
topologies, in that their Instance 1 traffic (human data) must be restricted to
1-hop from the root node to ensure the least possible delay in its delivery,
defeating the purpose of multi-hop mesh for critical communications.

The \gls{DTN} component using mobile drones as data mules provides an alternative
path for partitioned networks, with a maximum of 20 nodes per \gls{DODAG} to
maintain acceptable convergence times during emergency situations, but this
approach trades latency for eventual delivery rather than real-time mesh
routing.\\

These approaches reveal fundamental trade-offs: tree-based protocols like
\gls{RPL} offer structured routing for resource-constrained devices but create
single points of failure; fully distributed protocols like HyParView achieve
resilience through \gls{P2P} overlay management but assume network capabilities
impractical for embedded platforms; and hybrid multi-layer approaches
demonstrate practical deployment but still face throughput limitations.

\todo{FIXED: add take-away for the section + set up narrativo "motivates exploration..."}
This gap between resilience requirements and embedded capabilities
motivates the exploration of topology management strategies more suitable for
autonomous operation, particularly in unstable network conditions.

\subsection{Routing in Partitioned and Intermittently Connected Networks}
\label{subsec:routing_parts}

When network partitions prevent end-to-end paths, store-and-forward mechanisms
enable eventual data delivery. The Bundle Protocol~\cite{rfc5050} addresses
delay-tolerant networking through custody-based retransmission and
opportunistic connectivity exploitation. While the protocol specification
predates modern \gls{iot} deployments and was not designed specifically for
resource-constrained devices, its core principles inform contemporary \gls{DTN}
approaches.\\

The framework presented in~\cite{fast2018} implements elastic bandwidth
utilization by dynamically adjusting transmission rates based on available
connectivity, and supports scheduled, predicted, and opportunistic transmission
windows, taking inspiration from the Bundle Protocol.

Data parcels are compressed, encrypted, and bundled before transmission, with a
load balancer managing concurrent transfer threads to optimize bandwidth usage
during brief connectivity windows.

While their \gls{HTTP}-based implementation targets cloud-backed \gls{iot}
deployments, the core concepts of parceling data, maintaining transmission
queues, and opportunistic forwarding during connectivity windows translate to
\gls{P2P} scenarios where aggregation nodes become neighbors in a mesh network.

\subsection{Limitations of Centralized Coordination}
\label{subsec:limitations_centralized}

The work on Resilient Edge-enabled \gls{iot}~\cite{resilient2019} addresses
coordinator failures through dynamic leader election and backup mechanisms
within their framework.

Their coordination model divides environments into collaboration areas, with
resource-rich edge devices serving as coordinators that allocate tasks to
workers under their supervision when problems arise. Workers, in turn, are
\emph{active agents} such as robots and \gls{iot} devices that reside in a
particular environment, detect problems, and notify their coordinators.

When coordinators fail, the system automatically elects backups through
adaptive decentralized consensus, providing "gentle degradation" during
failures with restoration after recovery. Multiple coordinators operate
independently, eliminating single points of failure within the coordination
model itself.

This architecture depends fundamentally on edge servers running JVM-based SCAFI
middleware (a Scala library) to execute the aggregate programs that specify
coordination behavior, and these heavyweight infrastructure requirements --
both the Java runtime environment and resource-rich edge computing nodes --
conflict with embedded platform constraints and infrastructure-failure
scenarios.\todo{queremos mingle com o Babel... se calhar não entrar por aqui ou tudo bem porque não vamos usar os big-raspis para coordination?}

While the aggregate computing paradigm separates concerns (sensing, actuation,
communication, coordination), the implementation assumptions make it unsuitable
for disaster-resilient systems where edge servers may be the infrastructure
that fails. The formal guarantees of self-stabilization and compositional
properties come at the cost of persistent computational infrastructure that an
embedded-focused deployment cannot provide.

\subsection{Discussion}
\label{subsec:rw_discussion_2}

Existing peer discovery and topology mainteance approaches demonstrate
mechanisms appropriate for distributed operation but not without their
limitations for resource-constrained platforms aimed at disaster scenarios.

HyParView's~\cite{hyparview2007} hybrid view architecture (small active, large
passive) provides remarkable resilience to node failures, but the assumption of
routing and \gls{TCP} availability and persistent connections is not suitable
for embedded devices using connectionless radio technology or operating under
intermittent connectivity conditions.

The multi-protocol gateway's~\cite{multiproto_gateway}
\gls{BLE}-based neighbor discovery works on our targeted hardware but imposes
tree topologies with parent-child communication restrictions that limit route
diversity and create dependency chains.

\gls{RPL} routing in the heterogeneous disaster
architecture~\cite{heterogeneous-disaster} demonstrates practical WSN operation
but has significant throughput limitations: 20-second minimum send intervals at
10 hops, and restriction of critical traffic to 1-hop from root nodes.

The store-and-forward mechanism presented in the elastic bandwidth
framework~\cite{fast2018} enables partition tolerance but targets cloud-backed
deployments rather than \gls{P2P} mesh scenarios.

Coordination frameworks like the one in~\cite{resilient2019} provide dynamic
leader election but depend on JVM-based middleware on resource-rich edge
servers.\\

We have found no work that integrates connectionless neighbor discovery, hybrid
view maintenance, true P2P routing (not tree-based), and partition-tolerant
store-and-forward on resource-constrained embedded platforms.

\gls{ubabel}'s approach (Section \ref{component:p2p_mesh})
must adapt HyParView's architectural principles to \gls{BLE}/\gls{LoRa}
connectionless communication while supporting multi-hop routing without
depending on coordinators or routing infrastructure.
