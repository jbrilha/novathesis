%!TEX root = ../../template.tex

\gls{iot} and home automation (domotics) systems share the same technological
building blocks (wireless sensors, embedded devices, network communication) but
diverge significantly in their operational focus and architectures:

\todo{FIXED: added citations}
\begin{description}
    \item \textbf{\gls{iot} systems}~\cite{iottaxonomy2015} typically focus on
        \textbf{data collection and monitoring}, by streaming sensor readings
        to centralized platforms for analysis and processing, with control
        functions often being a secondary concern.
    \item \textbf{Domotics systems}~\cite{domotics2019,domoticstaxonomy2025}
        instead prioritize \textbf{real-time control and actuation} over
        physical spaces, where responsiveness and local autonomy are paramount
        for end user experience
        and privacy.
\end{description}

This distinction gains additional relevance when focusing on resilience
requirements: while \gls{iot} deployments may tolerate delayed data aggregation
and/or temporary connectivity issues in monitoring scenarios, domotics
applications (such as emergency lighting control or \gls{HVAC} management)
demand immediate local response regardless of network conditions.

Both domains, however, suffer from a common vulnerability when confronted with
infrastructure failures during disasters: their predominantly cloud-centric
architectures collapse precisely when autonomous operation becomes
indispensable.\\

The challenges faced by disaster-resilient \gls{iot} 
\todo{TOFIX: em vez de "iot systems":\\ smart systems?}
systems span multiple dimensions:
\begin{description}
    \item \textbf{Infrastructure failures} eliminate access points that devices
        depend on for coordination and operation;
    \item \textbf{Intermittent connectivity} creates network partitions where
        subgroups of devices must operate autonomously without global state
        synchronization;
    \item \textbf{Resource constraints} limit the computational, memory and
        energy budgets available for implementing sophisticated resilience
        mechanisms in embedded platforms.
\end{description}

These challenges are fundamentally architectural: existing \gls{iot} and
domotics systems are designed around the assumption of stable infrastructure,
treating network partitions and coordinator failures as transient anomalies
rather than the norm.

Cloud-centric architectures place control decisions in remote servers, creating
dependencies that cannot be satisfied when connectivity fails.
Coordinator-based topologies -- whether using dedicated gateways, master nodes,
\gls{SDN} controllers, among others -- concentrate the risk of failure in single
points that can paralyze entire systems if compromised or disconnected. Even
ostensibly distributed systems often rely on persistent connections (i.e.,
\gls{TCP}) over a single medium, or heavy middleware platforms that exceed the
capabilities of embedded devices and, once more, assume some form of
infrastructure availability.

The fundamental point of contention is between resilience requirements
(autonomy during infrastructure collapse) and resource constraints (limited
computation, memory, and energy). Traditional approaches address one at the
cost of the other: cloud platforms provide sophisticated coordination but fail
during outages; purely local systems avoid external dependencies but struggle
with inter-device coordination and protocol heterogeneity with low resource
usage.

Disaster-resilient systems require architectural choices that prioritize
\todo{TOFIX: general statement.. não encontro citations to back this up exactly, sinto que é kind of a given}
autonomous \gls{P2P} connectivity as the baseline mode of operation rather than
having it as a secondary -- even exceptional -- fallback mechanism.

Naturally, this requires a fundamental rethinking of \gls{iot} systems design:
how devices discover and communicate with each other using diverse protocol
stacks (without depending on centralized control), how they achieve temporal
synchronization and coordination without master beacons, how to optimize the
usage of limited available resources, and how to maintain secure operation
throughout without persistent access to inherently centralized components such
as \glspl{CA} or \glspl{KDC}.\\

The following sections briefly cover related work across four areas that
collectively enable autonomous operation: (Section~\ref{sec:wireless_comms})
common wireless communication technologies across application, link, and
physical layers; (Section~\ref{sec:p2p_mesh}) \gls{P2P} mesh networking and
topology management for autonomous network formation without coordinator
dependencies; (Section \ref{sec:multi_protocol}) multi-protocol communication
and adaptive selection based on \gls{QoS} and device capabilities;
(Section~\ref{sec:decentralized_sync}) decentralized synchronization mechanisms
for coordinating multi-protocol communication without coordinator nodes.

For each area, we examine how existing approaches handle (or fail to handle)
infrastructure failure scenarios, and identify architectural assumptions that
conflict with disaster-resilient requirements.

\todo{FIXED: separei a section sobre security, no longer part of "autonomous operation" areas per se}
Additionally, security and privacy considerations are presented in
Section~\ref{sec:security_scope}, which outlines the scope of this work in
regard to higher-layer safety mechanisms.

