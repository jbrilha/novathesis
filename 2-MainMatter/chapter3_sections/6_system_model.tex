%!TEX root = ../../template.tex

\section{System Model}
\label{sec:system_model}

\gls{ubabel} is structured as a layered architecture where components interact
to achieve autonomous multi-protocol operation. This section presents the key
architectural components organized into four functional layers: Core
Infrastructure, Discovery \& Topology, Coordination, and Data Management.

Figure~\ref{fig:system_architecture} \todo{FIGURA SEMELHANTE AO TRABALHO DO AKOS}
illustrates the overall component organization and interactions.

\subsection{Core Infrastructure Layer}
\label{subsec:core_infra}

\todo{mix entre o que temos e o que podemos vir a ter}
The Core Infrastructure Layer provides the foundational services for protocol
management, event dispatching, and connection handling. These components
abstract protocol heterogeneity and provide unified interfaces to higher
layers.

\subsubsection*{Protocol Manager}
\label{component:protocol_manager}

The Protocol Manager maintains runtime information for all available
communication protocols and device capabilities. It registers available
protocols during system initialization and tracks their operational status
throughout device operation.

Key responsibilities include:
\begin{itemize}
	\item Registering protocol handlers and their characteristics (range,
	      throughput, energy cost)
	\item Tracking which protocols are supported on each device based on
	      hardware capabilities
	\item Enabling or disabling protocols dynamically based on runtime
	      conditions (e.g., battery level, interference detection)
	\item Providing protocol capability information to the Protocol Selector
	      for adaptive selection decisions
\end{itemize}

% Key responsibilities include: (1) Registering protocol handlers and their
% characteristics (range, throughput, energy cost); (2) Tracking which protocols
% are supported on each device based on hardware capabilities; (3) Enabling or
% disabling protocols dynamically based on runtime conditions (e.g., battery
% level, interference detection); and (4) Providing protocol capability
% information to the Protocol Selector for adaptive selection decisions.

\subsubsection*{Event Dispatcher}
\label{component:event_dispatcher}

The Event Dispatcher implements an event-driven coordination model that
decouples components and enables asynchronous communication between services.
Components register interest in specific event types and subtypes, and the
dispatcher routes incoming events to all registered subscribers.

This approach prevents tight coupling between protocol implementations and
application logic, enabling independent development and testing of individual
components while maintaining system-wide coordination.

\subsubsection*{Communication Manager}
\label{component:communication_manager}

The Communication Manager provides protocol-agnostic connection state tracking
and high-level abstractions for peer communication. Applications and system
components interact with remote nodes using \glspl{UUID} rather than
protocol-specific addresses.

For each connection, the manager:
\begin{itemize}
	\item Handles connection establishment with remote peers across multiple
	      protocols
	\item Delivers messages to peers, via unicast, multicast or broadcast
	\item Tracks node state and connection integrity
\end{itemize}

This unified abstraction enables higher-layer components to interact with peers
without concerning themselves with protocol-specific details.

\subsection{Discovery \& Topology Layer}
\label{subsec:discovery_topology}
\todo{wishful thinking}

The Discovery \& Topology Layer handles peer discovery, network topology
maintenance, and multi-hop routing. These components enable autonomous network
formation without centralized coordination.

\subsubsection*{Peer Discovery Service}
\label{component:peer_discovery}

The Peer Discovery Service implements multi-protocol neighbor discovery.
Short-range protocols like \gls{BLE}, \gls{ESP-NOW}, and \gls{ZigBee} employ
active discovery mechanisms (i.e., advertising, peer lists, beacons), while
long-range \gls{LoRa} communication relies on opportunistic message exchange.

The service maintains a unified peer view by:
\begin{itemize}
	\item Periodically scanning for neighbor advertisements and beacons across
	      active protocols
	\item Advertising local device capabilities (supported protocols,
	      computational resources, battery status)
	\item Aggregating discovery information from multiple protocols into a
	      consistent peer database
\end{itemize}

Discovery handlers translate protocol-specific discovery information into
standardized peer descriptors, enabling the Topology Manager to operate on a
unified view regardless of the underlying medium.

\subsubsection*{Topology Manager}
\label{component:topology_manager}

Inspired by HyParView~\cite{hyparview2007}, the Topology Manager maintains
hybrid partial views for scalability and resilience. Each node keeps a small
\emph{active view} of peers with which it maintains active communication links,
and a larger \emph{passive view} serving as a pool of potential neighbors.

The active view is managed reactively: nodes are added during join operations
and removed upon failure detection. The passive view is maintained through
periodic shuffle operations that exchange peer descriptors with neighbors,
providing view diversity without global knowledge.

Key operations include:
\begin{itemize}
    \item Active view maintenance with symmetric link enforcement\todo{TBD.. not sure how feasible it is com lora/ble/esp now}
	\item Passive view shuffling for backup peer discovery
	\item Failure detection through connection timeout and explicit heartbeats
	\item Automatic peer promotion from passive to active view upon failures
\end{itemize}

\subsubsection*{Routing Service}
\label{component:routing}

The Routing Service enables multi-hop communication when direct peer
connectivity is unavailable. Unlike tree-based approaches
(Section~\ref{subsec:rw_discussion_2}) that restrict communication paths, the
Routing Service exploits the mesh topology maintained by the Topology Manager
to find alternative routes across differnet protocols.

Core functionality includes:
\begin{itemize}
	\item Multi-hop message forwarding using the active view
	\item Store-and-forward buffering for partitioned networks
	\item Path selection considering protocol characteristics and device
	      capabilities
	\item Opportunistic delivery when connectivity windows become available
\end{itemize}

When networks partition due to range limitations or failures, the Routing
Service maintains local buffers of undeliverable messages. Upon partition
reconciliation, buffered messages are opportunistically forwarded to their
destinations.

\subsection{Coordination Layer}
\label{subsec:coordination}
\todo{wishful thinking too}

The Coordination Layer manages protocol adaptation and time synchronization,
enabling coordinated operation without master nodes.

\subsubsection*{Protocol Selector}
\label{component:multi_protocol}

The Protocol Selector implements adaptive protocol selection based on message
type, device capabilities, and network conditions. Rather than using
fixed protocol assignments, it dynamically chooses communication technologies
considering multiple factors:

\begin{itemize}
    \item \textbf{\gls{QoS} requirements:} Latency-sensitive messages favor
        short-range high-throughput protocols (\gls{BLE}, \gls{ESP-NOW}), while
        delay-tolerant data can use long-range protocols (\gls{LoRa})
    \item \textbf{Device capabilities:} Protocol choices are constrained by
        both sender and receiver capabilities obtained from the Protocol
        Manager
	\item \textbf{Energy budget:} Low-battery devices prefer energy-efficient
	      protocols, potentially sacrificing throughput for lifetime
	\item \textbf{Protocol availability:} Interference or jamming may render
	      specific protocols unavailable, requiring an alternative selection
\end{itemize}

The selector tracks per-protocol switching costs to prevent excessive
oscillation between technologies, ensuring stable operation while remaining
adaptable.

\subsubsection*{Synchronization Service}
\label{component:decentralized_sync}

Coordinated multi-protocol operation requires time synchronization for channel
hopping and scheduled communication windows. The Synchronization Service
implements decentralized consensus on clock values through randomized gossip,
inspired by \gls{RGCS}~\cite{rgcs2018}.

Each node maintains a logical clock composed of rate and offset parameters that
transform its hardware clock into synchronized logical time. Synchronization
occurs through gossip exchanges using Poisson-distributed intervals to minimize
collision probability.
\todo{explicar melhor what I mean by Poisson-distributed?}

Key aspects include:
\begin{itemize}
	\item Per-protocol Poisson gossip rates accounting for protocol-specific
	      energy costs and reliability
	\item Converge-to-max criterion for faster convergence than average-based
	      approaches
	\item Simultaneous rate and offset compensation
	\item Multi-protocol operation where sync messages can be sent over any
	      available protocol
\end{itemize}

Unlike master-based synchronization that creates single points of failure, this
gossip-based approach maintains synchronization through distributed local
interactions.

\subsection{Data Management Layer}
\label{subsec:data_management}
\todo{wishful thinking too$^2$}

The Data Management Layer optimizes data transmission through compression and
efficient queue management, extending operational lifetime in battery-powered
deployments.

\subsubsection*{Compression Service}
\label{component:reduction_compression}

The Compression Service reduces transmission overhead through prediction-based
data reduction (inspired by Ambrosia~\cite{ambrosia2021}) and lightweight
compression (inspired by Sprintz~\cite{sprintz2018} and the approach
in~\cite{twotier2019}). It employs window-based forecasting where sensor values
are transmitted only when prediction error exceeds application-specific
thresholds.

Both sender and receiver maintain synchronized prediction state, ensuring
consistency without additional communication overhead. The service supports:

\begin{itemize}
    \item Prediction-based reduction with configurable error thresholds
        ($\delta$)
	\item Protocol-aware threshold adaptation (higher thresholds for
	      energy-expensive protocols)
	\item Lightweight time-series compression using delta coding, zigzag
	      encoding, and \gls{RLE}
	\item Graceful degradation under low battery conditions
\end{itemize}

Protocol-specific compression strategies account for varying energy costs: data
transmitted over \gls{LoRa} may employ more aggressive compression than
\gls{BLE} transmissions, balancing compression overhead against transmission
savings.

\subsubsection*{Message Queue Manager (??)}
\label{component:message_queue}

The Message Queue Manager handles priority-based queuing and buffer management
for messages awaiting transmission or forwarding. It maintains separate queues
for different message priorities (e.g., critical alerts, periodic sensor data,
opportunistic synchronization) and enforces memory limits based on device
capabilities.

Responsibilities include:
\begin{itemize}
    \item Priority queuing with configurable policies \inlinetodo{(strict priority, fair
        queuing, weighted fair queuing)??}
	\item Store-and-forward buffer management for partitioned networks
	\item Delivery confirmation tracking and retransmission scheduling
	\item Message aging awareness
\end{itemize}

During network partitioning, the queue manager coordinates with the Routing
Service to maintain buffered messages until connectivity is restored, subject
to available memory constraints.

\subsection{Component Interactions}
\label{subsec:component_interactions}

Figure~\ref{fig:component_interactions} illustrates the interactions betweencomponents.
\todo{INTERACTION DIAGRAM}

The Protocol Manager provides capability information to the Protocol Selector,
enabling adaptive selection decisions. Peer Discovery continuously feeds
neighbor information to the Topology Manager, which maintains active/passive
views and provides topology information to the Routing Service for multi-hop
forwarding decisions.

The Synchronization Service coordinates timing across components, particularly
ensuring the Protocol Selector does not switch protocols during
synchronization exchanges. The Compression Service reduces message size before
they reach the Communication Manager for transmission, while the Message Queue
Manager buffers messages and coordinates delivery with the Routing Service.

The Event Dispatcher orchestrates all inter-component communication, enabling
loosely coupled interactions through event publication and subscription. This
decoupling simplifies component development and incremental system evolution.
