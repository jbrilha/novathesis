%!TEX root = ../../template.tex

\section{System Model}
\label{sec:system_model}

\gls{ubabel} is structured as a layered architecture that provides foundational
services for distributed and autonomous communication using heterogeneous
wireless technologies. We distinguish between \emph{core components}, and
\emph{protocols}, where a protocol is a state machine implementing a
distributed algorithm, coordination mechanism, or application-specific logic.

Figure~\ref{fig:architecture_components} provides an overview of the system
architecture, with the core components that form the basis of the proposed
framework, and two complementary protocols geared towards autonomous peer
discovery and resilient topology management, explained in more detail below.

% trim=left bottom right top,
\begin{figure}[htbp]
	\centering
	\includegraphics[width=\linewidth]{2-MainMatter/chapter3_sections/arch_components.pdf}
	\caption{System architecture overview}
	\label{fig:architecture_components}
\end{figure}

\subsection{Core Components}
\label{subsec:core_components}

These provide the base runtime upon which all protocols operate. They handle
protocol registration and lifecycle management, event routing, wireless
communication abstractions, and time synchronization.

\paragraph{Protocol Manager}

Maintains a registry of active protocols and manages their
lifecycle throughout system operation, handling initialization,
resource allocation, and shutdown.

Key responsibilities include:
(1) Registering protocols and their associated event queues during initialization;
(2) Creating and managing \gls{RTOS} tasks for execution (one task per protocol);
(3) Tracking protocol state (running, suspended, stopped);
(4) Loading protocol configurations (i.e., from an SD card);
(5) Allocating and monitoring protocol resources;
(6) Providing protocol lookup services for directed message routing.

\paragraph{Event Dispatcher}
\label{component:event_dispatcher}

Implements an event-driven coordination model that decouples components and
enables asynchronous communication between protocols and services. The
dispatcher supports direct messaging between protocols and publish/subscribe
delivery.

Main responsibilities include:
(1) Managing component subscriptions to specific event types (i.e.,
notifications, timers, messages) and subtypes (protocol-specific);
(2) Broadcasting events to all registered subscribers for publish/subscribe
style communication;
(3) Routing directed messages to specific protocols using protocol identifiers
for point-to-point communication;
(4) Managing event memory lifecycle through reference counting as events are
delivered to multiple subscribers.

This enables system-wide notifications and targeted inter-protocol messages
while preventing tight coupling between components and application logic.

\paragraph{Communication Manager}
\label{component:communication_manager}

Manages the hybrid wireless capabilities of the device, handling
multiple communication technologies (Wi-Fi, \gls{BLE}, \gls{LoRa}, etc.)
and providing agnostic abstractions to higher layers.

Its main responsibilities include: (1) Establishing and maintaining (virtual)
connections to remote peers identified by \glspl{UUID}; (2) Detecting
connectivity events (i.e., new connections, disconnections) and notifying
interested components via the Event Dispatcher; (3) Transmitting outbound
messages (via unicast, multicast, or broadcast) over appropriate wireless
channels; (4) Receiving incoming messages and delivering them to the protocol
layer via the Event Dispatcher.

This component provides a unified interface for on-device protocols, allowing
higher layers to interact with mesh network participants via their \gls{UUID}
without concern for the underlying wireless technologies involved.

\paragraph{Time Synchronization Service}
\label{component:decentralized_sync}

Maintains a logical clock composed of rate and offset parameters that transform
a device's hardware clock into synchronized logical time, implementing
decentralized consensus on clock values through randomized gossip, inspired by
\gls{RGCS}~\cite{rgcs2018}.

Using Poisson-distributed intervals to minimize collision probability, it
randomly selects sync partners from active connections maintained by the
Communication Manager. Sync messages can be piggybacked on outgoing protocol
messages to reduce communication overhead.

Key aspects include:
(1) Technology-specific Poisson gossip rates accounting for energy costs and
reliability (e.g., different rates for \gls{BLE} vs. \gls{LoRa});
(2) Converge-to-max criterion for faster convergence than average-based approaches;
(3) Simultaneous rate and offset compensation;

Unlike coordinator-centric synchronization that creates single points of
failure, this gossip-based approach maintains synchronization through
distributed local interactions.

\subsection{User-Level Protocols}
\label{subsec:user_protos}

A user is, in effect, an application developer utilizing the core components of
\gls{ubabel}. Thus, we will provide two protocols out-of-the-box that will take
full advantage of and enhance the core components: one for peer discovery, and
another for topology management, which, when combined, enable autonomous mesh
operation without reliance on centralized coordination.

\paragraph{Peer Discovery Service}
\label{component:peer_discovery}

Implements neighbor discovery across different wireless technologies.
Short-range communication technologies like \gls{BLE}, \gls{ESP-NOW}, and
\gls{ZigBee} employ active discovery mechanisms (i.e., advertising, peer lists,
beacons), while long-range \gls{LoRa} communication relies on opportunistic
message exchange.

The service discovers peers and provides standardized information to the
Topology Manager by:
(1) Periodically scanning for neighbor advertisements and beacons;
(2) Advertising local device capabilities (supported technologies, computational resources, battery status);
(3) Translating technology-specific discovery information into standardized peer
descriptors for the Topology Manager.

\paragraph{Topology Manager}
\label{component:topology_manager}

Provides resilient overlay membership by maintaining hybrid partial views
(HyParView~\cite{hyparview2007}) of a network of devices, enabling the mesh to
survive infrastructure failures. 
It has two modes of operation: when infrastructure is available (e.g., Wi-Fi
access points), it builds and maintains an overlay topology that may span
beyond direct neighbors; when infrastructure fails, the topology gracefully
degrades to proximity-based membership using direct wireless links discovered
by the Peer Discovery Service.

It maintains a small \emph{active view} of peers for active communication, and
a larger \emph{passive view} of potential neighbors. The \emph{active view} is
managed reactively, such that nodes are added during join operations and
removed upon failure detection; and the \emph{passive view} is maintained
through periodic shuffle operations that exchange peer descriptors with
neighbors.

Key operations include:
(1) Building overlay topology leveraging available infrastructure;
(2) Periodic shuffling of passive view to maintain broader connectivity;
(3) Graceful degradation to proximity-based topology upon infrastructure failure;
(4) Automatic peer promotion from passive to active view for failover.

This protocol can serve multiple purposes: other protocols can query the active
view to select communication partners (e.g., the Synchronization Service
selecting gossip targets), the Communication Manager can prioritize maintaining
connections to active view members, and the passive view provides automatic
failover candidates when connections falter.
