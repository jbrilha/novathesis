%!TEX root = ../../template.tex

\section{System Model} % (fold)
\label{sec:system_model}

\subsection{Architecture}

The foundational building blocks of the \gls{ubabel} architecture can be
separated into the following \emph{components}, which aim to address each of
the aforementioned technical challenges.

\subsubsection*{Mesh Topology Manager}
\label{component:p2p_mesh}

As discussed in Section~\ref{sec:p2p_mesh}, tree topologies and
coordinator-based approaches limit resilience, and heavyweight middleware
platforms or HyParView's routing and \gls{TCP} requirements exceed embedded
device capabilities.

\todo{"temos ainda de endereçar o wireless"}
\gls{ubabel} intends to address these limitations through a fully
distributed \gls{P2P} mesh architecture without coordinator dependencies. The
system adapts HyParView's hybrid view concept (maintaining small active
neighbor sets for actual communication and larger passive backup lists for
failure recovery) but implements discovery and maintenance at the link layer
using \gls{BLE} and \gls{LoRa} advertisements rather than at the transport
layer via \gls{TCP} connections, accommodating the connectionless nature of
embedded radio/wireless communication.

Neighbor discovery is performed through advertising beacons emitted by one or
more of the supported protocols (e.g., \gls{BLE}, \gls{LoRa}, Wi-Fi) carrying
relevant peering information (node identity, capabilities, current neighbor
counts, etc.), with periodic gossip exchanges that propagate topology
information beyond single-hop range.

It's immediately apparent that -- if these exchanges aren't properly integrated
-- the same node advertising itself through different protocols could lead to
duplicate neighbor entries on the receiver side. To address this, nodes possess
unique IDs that are consistent across protocols, and by advertising their
capabilities they simultaneously inform potential neighbors of which protocols
they can expect to receive their advertisements from, which prevents
unnecessary processing of packets that would only contain redundant information
(e.g., by peeking at the sender's ID). Nodes supporting multiple protocols
naturally serve as bridges, enabling topology formation and message forwarding
between nodes that might not share the same protocol stack, this is exemplified
in Figure~\ref{fig:something}.
\todo{ADD FIGURE WITH (LORA ONLY) -> (BLE/LORA) -> (BLE ONLY)}

This approach mitigates potential range limitations: while \gls{BLE} typically
provides 10-50m coverage depending on environment and antenna configuration,
gossip-based propagation enables nodes to learn about distant neighbors through
multi-hop dissemination, supporting informed routing decisions and topology
adaptation.

Unlike the tree topology of Multi-Protocol Gateway or the \gls{RPL} \gls{DODAG}
structure of heterogeneous disaster architecture, \gls{ubabel} will
implement true \gls{P2P} mesh networking where any node can communicate with
any other node through dynamically selected multi-hop paths.

Route selection considers multiple factors gathered through local observation
and gossip: link quality metrics (\gls{RSSI}, \gls{PLR}), neighbor availability
across different protocols, and current energy budgets.

Store-and-forward mechanisms inspired by the Bundle Protocol will enable operation
during network partitions, but will be adapted for \gls{P2P} rather than
cloud-centric operation: data parcels are compressed (using techniques from
Section~\ref{sec:data_compression}), queued in local storage (flash, MicroSD
cards) when no forward path exists, and opportunistically transmitted when
connectivity windows emerge, be it through topology changes or protocol
switching creating new communication opportunities.

Coordination emerges from local interactions via gossip-based information
propagation, decentralized time synchronization
(Section~\ref{sec:decentralized_sync}), and adaptive protocol selection
(Section~\ref{sec:multi_protocol}), rather than centralized allocation or
scheduling of roles.

\subsubsection*{Multi-Protocol Communication Layer}
\label{component:multi_protocol}

While existing multi-channel approaches rely on centralized decision-making for
channel selection -- whether through master nodes (MCSC-WoT), \gls{SDN} controllers
(MINOS), or infrastructure endpoints -- \gls{ubabel} will address these
limitations through decentralized \emph{protocol} selection, extending beyond
single-medium channel hopping.

The system will employ opportunistic multi-protocol operation (\gls{BLE} + Wi-Fi +
\gls{LoRa} + \gls{ESP-NOW}) wherein devices adaptively select communication mediums based
on factors like message priority, neighbor availability, energy budget, network
conditions and device capabilities. This selection will be performed through
local decision-making informed by information gathered through gossip
protocols, without requiring \gls{SDN} controllers or infrastructure coordination.

\Gls{hysteresis} control mechanisms adapted from the Adaptive Protocol Selection
Framework prevent oscillations in these decisions while allowing rapid
response to changing conditions.

Protocol-specific compression strategies account for the different
energy/bandwidth trade-offs across the proposed stack (e.g. aggressive
compression for (relatively) energy-expensive \gls{LoRa}, lighter compression for
short-range \gls{BLE}), integrating the receiver energy asymmetry observations into
power management decisions.

Unlike AWCT's \gls{LoRaWAN} gateway dependency, Heterogeneous \gls{iot}'s
\gls{NVIS} centralized backhaul, MCSC-WoT's master-based synchronization, or
MINOS's \gls{SDN} controller requirement, \gls{ubabel} operates autonomously in
a \gls{P2P} manner with no static coordinator dependencies.

The multi-protocol capability provides resilience through diversity rather than
optimization through centralized selection: when conditions render one medium
unsuitable, devices autonomously transition to alternatives without requiring
coordinator intervention.

\subsubsection*{Decentralized Synchronization Service}
\label{component:decentralized_sync}

While MCSC-WoT demonstrates master-based synchronization feasibility and
\gls{RGCS} provides masterless convergence, both assume homogeneous
single-protocol networks.

\gls{ubabel} will extend \gls{RGCS} to heterogeneous multi-protocol
networks through protocol-specific Poisson processes with distinct activation
rates. Rather than a single $\lambda$ controlling overall sync frequency, the
system employs per-protocol gossip rates that take into account the energy
required for each transmission, and the current network conditions:

\begin{description}
	\item \textbf{$\lambda$\_\gls{BLE}} (high frequency): provides rapid local
	      synchronization with nearby neighbors, leveraging \gls{BLE}'s low energy cost
	      for frequent exchanges
	\item \textbf{$\lambda$\_\gls{LoRa}} (low frequency): maintains long-range
	      temporal coordination, using \gls{LoRa}'s more expensive transmissions
	      sparingly
	\item \textbf{$\lambda$\_Wi-Fi} / Others (medium frequency): situational
	      use based on network density, energy budgets and other factors
\end{description}

Each protocol will have its own Poisson-triggered gossip loop, but all updates
contribute to a single shared logical clock via the converge-to-max criterion
explored above. When sync events from different protocols occur
near-simultaneously, the max operation's associativity ensures correct
convergence regardless of update order.

This multi-graph gossip approach provides several advantages over homogeneous
sync:
\begin{description}
	\item \textbf{Faster convergence:} High-frequency \gls{BLE} gossip can achieve
	      tighter local consensus while sparse \gls{LoRa} exchanges prevent drift
	      between distant clusters. The combination converges faster than either
	      protocol alone.
	\item \textbf{Resilience through diversity:} If 2.4 GHz interference
	      disrupts \gls{BLE}/Wi-Fi synchronization, \gls{LoRa} maintains loose temporal
	      coordination across the network. Conversely, if \gls{LoRa} experiences poor
	      conditions, local \gls{BLE} sync keeps nearby nodes coordinated.
	\item \textbf{Natural energy optimization:} Protocol characteristics
	      directly inform sync frequencies, so that more expensive long-range
	      transmissions occur rarely while cheap short-range exchanges happen
	      frequently, matching energy budgets to communication needs.
\end{description}

The \gls{hysteresis} control mechanisms adapted from
Section~\ref{sec:multi_protocol} will prevent oscillations in protocol
selection during sync events: sync partner and protocol choices stabilize
around locally optimal configurations rather than continuously switching
between equally-viable options.

The approach aims to handle network partitions gracefully: nodes maintain
synchronization within their reachable partition using available protocols, and
when partitions merge (through mobility or topology changes), the
converge-to-max criterion naturally reconciles previously independent temporal
references without requiring special merge logic.

Unlike MCSC-WoT's master-based approach, this architecture will operate in a
fully \gls{P2P} fashion with no fixed coordinator dependencies. The
multi-protocol extension goes beyond \gls{RGCS}'s original homogeneous network
assumptions, since heterogeneous gossip graphs with protocol-specific
characteristics warrant distinct sync frequencies, enabling simultaneous
optimization of convergence speed, energy consumption, and range coverage.

\subsubsection*{Data Reduction and Compression Module}
\label{component:reduction_compression}
\todo{maybe not worth defining as a ""component"" per se}

Extending Ambrosia's lightweight prediction approach to multi-protocol
scenarios, \gls{ubabel} will reduce data transmission through
prediction-based filtering: nodes transmit sensor readings only when prediction
error exceeds configurable thresholds (similar to Ambrosia), with
protocol-aware tuning to balance accuracy against energy constraints across
heterogeneous protocols.

Building on prediction-based reduction, the goal is to apply Delta+\gls{RLE} compression
to the readings that are transmitted, combining both techniques for added
energy savings and reduced data volume over the air.

Following the two-tier architecture pattern, compression occurs at sensor nodes
before transmission, with just the first sample in each window being
transmitted with no compression applied.

Adaptive threshold selection will adjust $\delta$ based on both channel
characteristics and node state:
\todo{estes deltas vinham do ambrosia, fica confuso agora que falo de delta encoding?}
\begin{itemize}
	\item \textbf{High $\delta$ for \gls{LoRa}}: Tolerates larger prediction errors
	      to minimize expensive long-range transmissions, sending only when
	      predictions deviate significantly. These packets undergo
	      aggressive Delta+\gls{RLE} compression to minimize airtime costs.
	\item \textbf{Low $\delta$ for \gls{BLE}}: Requires tighter prediction accuracy
	      for cheap local exchanges, allowing for more frequent transmissions to
	      maintain precision. Lighter compression (delta encoding only)
	      is applied here for lower latency.
	\item \textbf{Battery-aware adaptation}: As node energy depletes, increased
	      $\delta$ across all channels reduces transmission frequency, extending
	      lifetime for critical messages. Compression at this point becomes more
	      lossy to further minimize transmitted data volume.
\end{itemize}

\textbf{Application-aware transmission} will distinguish between data types:
periodic sensor readings (temperature, humidity, light) use window-based
prediction with configurable $\delta$ thresholds, while discrete events
(emergency alerts, button presses, motion detection) transmit immediately
without prediction. For predictable streams, accuracy-critical applications
maintain low $\delta$, while trend monitoring can accept higher values to
reduce transmission frequency.

Compression strategies also vary by data type: environmental sensors can
tolerate more prominent Delta+\gls{RLE} compression given their high temporal
correlation, while critical alerts transmit with minimal (if any) compression
to reduce latency.

\textbf{Gateway energy management} accounts for the 15-20\% receiver energy
cost by implementing selective radio shutdown: gateways can disable specific
protocols during low-activity periods, waking periodically to check for
incoming sync beacons or when prompted via other protocols. Decompression
overhead in these nodes is minimal compared to transmission costs at
the sensor level.
\todo{entrar pela coisa de desligar os rádios ainda mais?}

The expectation is that the combination of lightweight prediction, adaptive
thresholding, and protocol-specific tuning together with Delta+\gls{RLE} compression
will provide substantial energy savings without requiring complex and
computationally intensive compression algorithms that would themselves consume
significant power.

% \subsubsection*{Security and Key Management}
% \label{component:security_keys}
%
% While zero-config approaches face circular authentication dependencies and
% centralized schemes fail when infrastructure collapses, \gls{ubabel}
% will adopt a hybrid security model that distinguishes between pre-disaster
% deployment and post-disaster autonomous operation.
%
% \begin{description}
% 	\item \textbf{Pre-disaster deployment phase:} For planned installations
% 	      (building sensor networks, campus-wide monitoring), devices are
% 	      provisioned during initial deployment:
%
% 	      \begin{itemize}
% 		      \item \textbf{LPKM polynomial shares:} Preloaded via \gls{KDC} for
% 		            distributed key management
% 		      \item \textbf{GASE secret-shadows:} Two per node for threshold
% 		            authentication ($(t-1)$-of-$n$ reveal protocol)
% 	      \end{itemize}
%
% 	      These are reasonable assumptions for disaster-resilience scenarios, as
% 	      preparation happens before any emergency operation needs to take place.
%
% 	\item \textbf{Post-disaster autonomous operation:} Once a disaster occurs
% 	      and infrastructure fails, the network operates autonomously using
% 	      mechanisms that require no central coordination:
%
% 	      \begin{itemize}
% 		      \item \textbf{Threshold-based authentication} (adapted from
% 		            \gls{GASE}): Nodes form ad-hoc authentication groups where
% 		            $(t-1)$-of-$n$ members reveal secret shares (pre-distributed)
% 		            to derive session keys via Lagrange interpolation. Local
% 		            coordinator election replaces GASE's centralized \glspl{GL}
% 		            Session key derivation combines group secrets with
% 		            device-specific keys to prevent impersonation, while
% 		            aggregated \gls{MAC} tags enable efficient multi-node
% 		            verification without individual authentication overhead.
% 		            \todo{too dense?}
% 		      \item \textbf{Distributed revocation} (adapted from \gls{LPKM}):
% 		            nodes independently compute updated keys excluding
% 		            compromised peers, with revocation decisions propagated via
% 		            gossip
% 		      \item \textbf{Periodic share updating} (adapted from \gls{LPKM}):
% 		            timer-based key refresh for backward secrecy, no coordinator
% 		            needed
% 		      \item \textbf{Lightweight encryption}: \gls{AES} encryption
% 		            (demonstrated by MCSC-WoT on ESP32) with
% 		            compress-then-encrypt strategies
% 		            (Section~\ref{sec:data_compression}) to minimize ciphertext
% 		            size and transmission energy
% 		      \item \textbf{Channel hopping/protocol switching for security}:
% 		            multi-channel/multi-protocol operation complicates
% 		            eavesdropping and uses masterless synchronization
% 		            (\gls{RGCS}) rather than master-based beacons
% 	      \end{itemize}
% \end{description}
