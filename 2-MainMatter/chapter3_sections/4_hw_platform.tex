%!TEX root = ../../template.tex

\section{Hardware Platform}
\label{sec:hw_platform}

\gls{ubabel} targets resource-constrained embedded platforms commonly used in
\gls{iot} and domotics systems. The following sections cover device
specifications and communication protocol characteristics, establishing the
foundation for our multi-protocol, heterogeneous architecture.

We distinguish between two device classes based on their capabilities and roles
within the architecture:

\paragraph{Class-1: Battery-Powered Sensor Nodes:} The primary targets for
\gls{ubabel}. These embedded devices handle sensing, actuation, \gls{P2P}
communication, and decentralized coordination without requiring external
infrastructure. \gls{ubabel} operates autonomously with Class-1 devices alone.

\paragraph{Class-2: Optional Aggregation Nodes:} Resource-rich devices that can
enhance deployments with data aggregation, and optional cloud bridging when
infrastructure is available. These devices run the broader
Babel~\cite{fouto2022babel} framework and are mentioned here for completeness,
but are not required for \gls{ubabel}'s core autonomous operation.

\subsection{Class-1: Sensor Nodes}

Table~\ref{tab:c1_device_specs} presents the resources available across Class-1
devices. These span from basic sensor nodes (Pico 2W, ESP32-C5/C6) to more
capable embedded devices (ESP32-S3/P4), enabling role differentiation
within the autonomous peer network.

\todo{figure out table spacing issues }
\rowcolors{1}{}{GhostWhite}
\begin{table}[htbp]
	\centering
	\caption{Class-1 Device Specifications}
	\label{tab:c1_device_specs}
	\footnotesize
	\begin{tabular}{@{}l p{5.2cm} c c c c @{}}
		\toprule
		\rowcolor{Gainsboro}%
		\textbf{Device}       & \textbf{\glslink{CPU}{CPU}$^1$} & \textbf{CPU Clock} & \textbf{\glslink{SRAM}{SRAM}} & \textbf{\glslink{PSRAM}{PSRAM}$^2$} & \textbf{Flash} \\
		\midrule
		Raspberry Pi Pico 2 W & Dual-Core Arm Cortex-M33        & 150 MHz            & 520 KB                        & \xmark                              & 4 MB           \\
		ESP32                 & Dual-Core Xtensa LX6            & 240 MHz            & 520 KB                        & \xmark                              & 16 MB          \\
		ESP32-C5              & Single-Core RISC-V              & 240 MHz            & 384 KB                        & \xmark                              & 4 MB           \\
		ESP32-C6              & Single-Core RISC-V              & 160 MHz            & 512 KB                        & \xmark                              & 4 MB           \\
		ESP32-S3              & Dual-Core Xtensa LX7            & 240 MHz            & 512 KB                        & 8 MB                                & 16 MB          \\
		ESP32-P4              & Dual-Core RISC-V                & 360 MHz            & 768 KB                        & 32 MB                               & 32 MB          \\
		\bottomrule
		\multicolumn{6}{l}{\footnotesize $^1$ All supported devices are 32-bit}                                                                                             \\
		\multicolumn{6}{p{\linewidth}}{\footnotesize
			$^2$ \gls{PSRAM} is supported on all devices, but not available in the particular models used during development
		}                                                                                                                                                                   \\
	\end{tabular}
\end{table}

Resource availability within Class-1 devices varies significantly both in
processing power and available memory. While there are no fixed roles,
capability awareness naturally influences dynamic selection procedures:

\paragraph{Group Leader Preference:} Higher-resource devices (ESP32-S3/P4) are
prioritized in local \gls{GL} elections due to their ability to efficiently
aggregate member \glspl{MAC} and maintain group state
(Section~\ref{component:security_keys}). Resource availability merely
influences preference, not eligibility.

\paragraph{Protocol Bridging:} Devices with broader protocol support naturally
serve as bridges between protocol domains, forwarding messages across different
mediums based on advertised capabilities.\\

\todo{not sure how pico handles it, tenho de confirmar se isto é verdade}
All Class-1 devices support flash encryption for cryptographic primitive
storage.

\subsubsection*{Protocol Diversity and Device Coverage}

Table~\ref{tab:device_protos} shows the communication protocols supported
across Class-1 devices, demonstrating considerable heterogeneity.

\rowcolors{1}{}{GhostWhite}
\begin{table}[htbp]
	\centering
	\caption{Supported Communication Protocols}
	\label{tab:device_protos}
	\footnotesize
	\begin{tabular}{@{}lccccc@{}}
		\toprule
		\rowcolor{Gainsboro}%
		\textbf{Device} & \textbf{Wi-Fi} & \textbf{Bluetooth}     & \textbf{\gls{ESP-NOW}} & \textbf{\gls{ZigBee}} & \textbf{\gls{LoRa}$^1$} \\
		\midrule
		Pico 2W         & 2.4GHz         & Classic + \gls{LE} 5.2 & \xmark                 & \xmark                & \checkmark              \\
		ESP32           & 2.4 GHz        & Classic + \gls{LE} 4.2 & \checkmark             & \xmark                & \checkmark              \\
		ESP32-C5$^2$    & 2.4/5 GHz      & \gls{LE} 6.0           & \checkmark             & 3.0                   & \checkmark              \\
		ESP32-C6$^2$    & 2.4 GHz        & \gls{LE} 5.3           & \checkmark             & 3.0                   & \checkmark              \\
		ESP32-S3        & 2.4 GHz        & \gls{LE} 5.0           & \checkmark             & \xmark                & \checkmark              \\
		ESP32-P4$^3$    & 2.4 GHz        & \gls{LE} 5.3           & \checkmark             & \xmark                & \checkmark              \\
		\bottomrule
		\multicolumn{6}{p{\linewidth}}{\footnotesize
			$^1$ Via external SX1276/SX1262 transceiver, controlled through
			\gls{SPI}
		}                                                                                                                                    \\
		\multicolumn{6}{p{\linewidth}}{\footnotesize
			$^2$ ESP32-C5/C6 also support 802.11ax (Wi-Fi 6) and \gls{Thread},
			but we focus on \gls{ZigBee} given its wider adoption in \gls{iot}
			systems
		}                                                                                                                                    \\
		\multicolumn{6}{p{\linewidth}}{\footnotesize
			$^3$ ESP32-P4 does not possess wireless capabilities by itself and
			relies on a C6-Mini coprocessor, with ZigBee support still in
			development
		}                                                                                                                                    \\
	\end{tabular}
\end{table}

Protocol support across devices provides several advantages:

\paragraph{Coverage:} All devices support at least three protocols (Wi-Fi,
\gls{BLE}, \gls{LoRa}), ensuring baseline multi-protocol operation across the
entire platform range. Even the least capable device (Pico 2 W) can participate
in protocol-diverse mesh networks.

\paragraph{Protocol-Specific Advantages:} ESP32 family devices can leverage
\gls{ESP-NOW} for \gls{P2P} communication without infrastructure. ESP32-C5/C6
additionally provide \gls{ZigBee} support for standardized mesh networking --
particularly interesting in building automation and domotics contexts.

Notably, ESP32-C5 supports \gls{ESP-NOW} over both 2.4 and 5 GHz
bands, which provides a crucial short-range alternative path for congested
environments where \gls{BLE}, Wi-Fi and \gls{ZigBee} all contend for 2.4 GHz
channels.

\paragraph{Long-Range Capabilities:} All platforms support external \gls{LoRa}
transceivers via \gls{SPI}, enabling long-range communication for disaster
scenarios. This addresses the multi-protocol resilience requirements discussed
in Section~\ref{sec:multi_protocol}.\\

These factors combine to provide multi-protocol resilience: spectrum diversity
across 2.4/5 GHz and sub-GHz bands, infrastructure-free operation via
\gls{ESP-NOW} and \gls{BLE}, and long-range connectivity through \gls{LoRa}.

% \subsubsection*{Protocol Characteristics}
%
% Table~\ref{tab:protocol_specs} details the basic operational characteristics of
% each of the supported protocols.
%
% \todo{get actual power draw values from somewhere}
%
% \rowcolors{1}{}{GhostWhite}
% \begin{table}[htbp]
% 	\centering
% 	\caption{Protocol Characteristics}
% 	\label{tab:protocol_specs}
% 	\footnotesize
% 	\begin{tabular}{@{}lccccc@{}}
% 		\toprule
% 		\rowcolor{Gainsboro}%
% 		\textbf{Protocol}  & \textbf{Discovery}        & \textbf{Range$^1$} & \textbf{Frequency}  & \textbf{Data Rate} & \textbf{Power Draw} \\
% 		\midrule
% 		Wi-Fi              & \glslink{SSID}{SSID} scan & 50-100 m           & 2.4 GHz             & 1-50 Mbps          & High                \\
% 		\glslink{BLE}{BLE} & Advertising               & 10-50 m            & 2.4 GHz             & 1-2 Mbps           & Low                 \\
% 		\gls{ESP-NOW}      & Peer list                 & 50-100 m           & 2.4 GHz             & ~1 Mbps            & Very Low            \\
% 		\gls{ZigBee}       & Beacon                    & 10-100 m           & 2.4 GHz             & 250 Kbps           & Very Low            \\
% 		\gls{LoRa}         & Preamble detection        & 2-10 km            & 433/868/915 MHz$^2$ & 0.3-50 Kbps$^3$    & Low$^3$             \\
% 		\bottomrule
% 		\multicolumn{6}{p{\linewidth}}{\footnotesize
% 			$^1$ Outdoor line-of-sight; indoor ranges typically 30-50\% lower
% 		}                                                                                                                                    \\
% 		\multicolumn{6}{p{\linewidth}}{\footnotesize
% 			$^2$ Frequency depends on regional regulations
% 		}                                                                                                                                    \\
% 		\multicolumn{6}{p{\linewidth}}{\footnotesize
% 			$^3$ Data rate and power consumption depend on \gls{SF}; higher \gls{SF} = longer range, lower rate, higher energy cost
% 		}                                                                                                                                    \\
% 	\end{tabular}
% \end{table}
%
% These characteristics drive protocol selection based on deployment context and
% communication requirements:
%
% \paragraph{Range vs. Throughput Trade-offs:} \gls{LoRa} provides 2-10 km range
% (with appropriate antenna configurations) but very limited throughput (0.3-50
% Kbps), suitable for sparse periodic synchronization between distant partitions.
% \gls{BLE} and \gls{ESP-NOW} offer moderate range (50-100 m) with higher
% throughput (1-2 Mbps), appropriate for dense local deployments where multi-hop
% forwarding is feasible and often required.
%
% \paragraph{Discovery Mechanisms:} Different protocols employ distinct peer
% discovery approaches. \gls{BLE} advertising enables continuous passive
% discovery, \gls{ESP-NOW} requires explicit peer lists, and \gls{ZigBee} uses
% beacon-based association. \gls{ubabel}'s hybrid discovery mechanism
% (Section~\ref{component:multi_protocol}) will accommodate these differences
% while maintaining a unified peer view across protocols.
%
% \paragraph{Energy Considerations:} Protocol selection must account for energy
% costs: Wi-Fi consumes significantly more power than \gls{BLE}, while
% \gls{LoRa} power consumption depends heavily on \gls{SF} configuration. The
% data compression mechanisms (Section~\ref{component:reduction_compression})
% interact with protocol selection to minimize overall energy expenditure.\\
%
\todo{REWORD THIS SINCE IT WAS MOVED}
Together, these device capabilities and protocol characteristics establish the
hardware foundation for \gls{ubabel}'s infrastructure-independent operation.
The following section details how these resources are orchestrated through
\gls{ubabel}'s architectural approach.

\subsection{Class-2: Aggregation Nodes}

For deployments where additional resources are available, Class-2 devices can
enhance \gls{ubabel} networks without introducing coordinator dependencies.
Table~\ref{tab:c2_device_specs} presents the resources available to these
devices.

\rowcolors{1}{}{GhostWhite}
\begin{table}[htbp]
	\centering
	\caption{Class-2 Device Specifications}
	\label{tab:c2_device_specs}
	\footnotesize
	\begin{tabular}{@{}l p{5.2cm} c c c @{}}
		\toprule
		\rowcolor{Gainsboro}%
		\textbf{Device} & \textbf{\glslink{CPU}{CPU}$^1$} & \textbf{CPU Clock} & \textbf{\glslink{RAM}{RAM}}$^1$ & \textbf{Storage}  \\
		\midrule
		Raspberry Pi 5  & Quad-Core 64-bit Arm Cortex-A76 & 2.4 GHz            & 4-16 GB                         & Variable external \\
		Raspberry Pi 4  & Quad-Core 64-bit Arm Cortex-A72 & 1.8 GHz            & 4-8 GB                          & Variable external \\
		\bottomrule
		\multicolumn{5}{p{\linewidth}}{\footnotesize
			$^1$ Configurations with less \gls{RAM} are available but were not considered
		}                                                                                                                            \\
	\end{tabular}
\end{table}

Class-2 devices serve as \emph{optional} infrastructure that, when present,
can perform:
\begin{itemize}
	\item Data aggregation and long-term storage
	\item Computationally intensive analytics beyond Class-1 capabilities
	\item Optional cloud bridging when external connectivity is available and
	      desired (e.g., \gls{MQTT} brokerage, \gls{CoAP} endpoints)
\end{itemize}

Critically, Class-2 devices \emph{do not} serve as coordinators or single
points of failure. \gls{ubabel} networks will operate fully autonomously with
Class-1 devices alone; Class-2 nodes participate as peers that happen to have
greater resources, not as masters. Their absence does not prevent network
formation, peer discovery, or autonomous operation.

