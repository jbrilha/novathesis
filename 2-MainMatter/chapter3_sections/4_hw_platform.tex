%!TEX root = ../../template.tex

\section{Hardware Platform}
\label{sec:hw_platform}

\gls{ubabel} targets resource-constrained embedded platforms commonly used in
\gls{iot} and domotics systems. The following sections cover device
specifications and radio technology characteristics, establishing the
foundation for our hybrid architecture.

We distinguish between two device classes based on their capabilities and roles
within the architecture.

\subsection{Class-1: Battery-Powered Sensor Nodes}

The primary targets for \gls{ubabel}. These  embedded devices
handle sensing, actuation, \gls{P2P} communication, and decentralized
coordination without requiring external infrastructure.

Table~\ref{tab:c1_device_specs} presents the resources available across Class-1
devices. These span from basic sensor nodes (Pico 2W, ESP32-C5/C6) to more
capable embedded devices (ESP32-S3/P4), enabling role differentiation
within the autonomous peer network.

\todo{TODO: figure out table spacing}
\rowcolors{1}{}{GhostWhite}
\begin{table}[htbp]
	\centering
	\caption{Class-1 Device Specifications}
	\label{tab:c1_device_specs}
	\footnotesize
	\begin{tabular}{@{}l p{5.2cm} c c c c @{}}
		\toprule
		\rowcolor{Gainsboro}%
		\textbf{Device}       & \textbf{\glslink{CPU}{CPU}$^1$} & \textbf{CPU Clock} & \textbf{\glslink{SRAM}{SRAM}} & \textbf{\glslink{PSRAM}{PSRAM}$^2$} & \textbf{Flash} \\
		\midrule
		Raspberry Pi Pico 2 W & Dual-Core Arm Cortex-M33        & 150 MHz            & 520 KB                        & \xmark                              & 4 MB           \\
		ESP32                 & Dual-Core Xtensa LX6            & 240 MHz            & 520 KB                        & \xmark                              & 16 MB          \\
		ESP32-C5              & Single-Core RISC-V              & 240 MHz            & 384 KB                        & \xmark                              & 4 MB           \\
		ESP32-C6              & Single-Core RISC-V              & 160 MHz            & 512 KB                        & \xmark                              & 4 MB           \\
		ESP32-S3              & Dual-Core Xtensa LX7            & 240 MHz            & 512 KB                        & 8 MB                                & 16 MB          \\
		ESP32-P4              & Dual-Core RISC-V                & 360 MHz            & 768 KB                        & 32 MB                               & 32 MB          \\
		\bottomrule
		\multicolumn{6}{l}{\footnotesize $^1$ All supported devices are 32-bit}                                                                                             \\
		\multicolumn{6}{p{\linewidth}}{\footnotesize
			$^2$ \gls{PSRAM} is supported on all devices, but not available in the particular models used during development
		}                                                                                                                                                                   \\
	\end{tabular}
\end{table}

Resource availability within Class-1 devices varies significantly both in
processing power and available memory. While there are no fixed roles,
capability awareness naturally influences dynamic selection procedures:

\paragraph{Data Aggregation and Buffering:} Higher-resource devices
(ESP32-S3/P4) can maintain larger message buffers and aggregate data from
multiple sensor nodes, which is particularly valuable during network partitions
when store-and-forward mechanisms come into effect.

\paragraph{Interface Bridging:} Devices with broader communication technology
support naturally serve as bridges between domains, forwarding messages across
different mediums based on advertised capabilities.\\

\subsubsection*{Radio Interface Diversity and Device Coverage}

Table~\ref{tab:device_radio_tech} shows the different radio technologies
supported across Class-1 devices, demonstrating considerable heterogeneity.

\rowcolors{1}{}{GhostWhite}
\begin{table}[htbp]
	\centering
	\caption{Supported Wireless Technologies}
	\label{tab:device_radio_tech}
	\footnotesize
	\begin{tabular}{@{}lccccc@{}}
		\toprule
		\rowcolor{Gainsboro}%
		\textbf{Device} & \textbf{Wi-Fi} & \textbf{Bluetooth}     & \textbf{\gls{ESP-NOW}} & \textbf{\gls{ZigBee}} & \textbf{\gls{LoRa}$^1$} \\
		\midrule
		Pico 2W         & 2.4GHz         & Classic + \gls{LE} 5.2 & \xmark                 & \xmark                & \checkmark              \\
		ESP32           & 2.4 GHz        & Classic + \gls{LE} 4.2 & \checkmark             & \xmark                & \checkmark              \\
		ESP32-C5$^2$    & 2.4/5 GHz      & \gls{LE} 6.0           & \checkmark             & 3.0                   & \checkmark              \\
		ESP32-C6$^2$    & 2.4 GHz        & \gls{LE} 5.3           & \checkmark             & 3.0                   & \checkmark              \\
		ESP32-S3        & 2.4 GHz        & \gls{LE} 5.0           & \checkmark             & \xmark                & \checkmark              \\
		ESP32-P4$^3$    & 2.4 GHz        & \gls{LE} 5.3           & \checkmark             & \xmark                & \checkmark              \\
		\bottomrule
		\multicolumn{6}{p{\linewidth}}{\footnotesize
			$^1$ Via external SX1276/SX1262 transceiver, controlled through
			\gls{SPI}
		}                                                                                                                                    \\
		\multicolumn{6}{p{\linewidth}}{\footnotesize
			$^2$ ESP32-C5/C6 also support 802.11ax (Wi-Fi 6) and \gls{Thread},
			but we focus on \gls{ZigBee} given its wider adoption in \gls{iot}
			systems
		}                                                                                                                                    \\
		\multicolumn{6}{p{\linewidth}}{\footnotesize
			$^3$ ESP32-P4 does not possess wireless capabilities by itself and
			relies on a C6-Mini coprocessor, with ZigBee support still in
			development
		}                                                                                                                                    \\
	\end{tabular}
\end{table}

The wide variety of communication support across devices provides several
advantages:

\paragraph{Coverage:} All devices support at least three radio communication
technologies (Wi-Fi, \gls{BLE}, \gls{LoRa}), ensuring baseline hybrid operation
across the entire platform range, such that even the least capable device (Pico
2 W) can participate in heterogeneous mesh networks. Particularly, \gls{LoRa}
is supported using external transceivers via \gls{SPI}, ensuring long-range
connectivity across the board.

\paragraph{Device-Specific Advantages:} ESP32 family devices can leverage
\gls{ESP-NOW} for \gls{P2P} communication without infrastructure. ESP32-C5/C6
additionally provide \gls{ZigBee} support for standardized mesh networking --
particularly interesting in building automation and domotics contexts.

Notably, ESP32-C5 supports \gls{ESP-NOW} over both 2.4 and 5 GHz
bands, which provides a crucial short-range alternative path for congested
environments where \gls{BLE}, Wi-Fi and \gls{ZigBee} all contend for 2.4 GHz
channels.\\

These factors combine to provide remarkable resilience: spectrum diversity
across 2.4/5 GHz and sub-GHz bands, infrastructure-free operation via
\gls{ESP-NOW} and \gls{BLE}, and long-range connectivity through \gls{LoRa}.

Together, these Class-1 specifications and radio capabilities establish the
core hardware foundation for \gls{ubabel}'s autonomous operation.
Section~\ref{sec:system_model} details how these resources are exploited in the
proposed architecture.

\subsection{Class-2: Optional Aggregation Nodes}

Resource-rich devices that run the broader Babel~\cite{fouto2022babel}
framework and are mentioned here for the sake of integration with broader
systems, but are not required for \gls{ubabel}'s core autonomous operation.

This class encompasses general-purpose computing devices such as desktops,
laptops, and servers. For development purposes, and staying within the realm
of embedded platforms, we specifically utilize Raspberry Pi \glspl{SBC},
whose specifications are available in Table~\ref{tab:c2_device_specs}.

\rowcolors{1}{}{GhostWhite}
\begin{table}[htbp]
	\centering
	\caption{Class-2 Device Specifications}
	\label{tab:c2_device_specs}
	\footnotesize
	\begin{tabular}{@{}l p{5.2cm} c c c @{}}
		\toprule
		\rowcolor{Gainsboro}%
		\textbf{Device} & \textbf{\glslink{CPU}{CPU}$^1$} & \textbf{CPU Clock} & \textbf{\glslink{RAM}{RAM}}$^1$ & \textbf{Storage}  \\
		\midrule
		Raspberry Pi 5  & Quad-Core 64-bit Arm Cortex-A76 & 2.4 GHz            & 4-16 GB                         & Variable external \\
		Raspberry Pi 4  & Quad-Core 64-bit Arm Cortex-A72 & 1.8 GHz            & 4-8 GB                          & Variable external \\
		\bottomrule
		\multicolumn{5}{p{\linewidth}}{\footnotesize
			$^1$ Configurations with less \gls{RAM} are available but were not considered
		}                                                                                                                            \\
	\end{tabular}
\end{table}

Class-2 devices serve as \emph{optional} infrastructure that, when present, can
perform: (1) Data aggregation and long-term storage; (2) Computationally
intensive analytics beyond Class-1 capabilities; and (3) Optional cloud
bridging when external connectivity is available and desired (e.g., \gls{MQTT}
brokerage, \gls{CoAP} endpoints).

Critically, these devices \emph{do not} serve as coordinators or single points
of failure. \gls{ubabel} networks will operate fully autonomously with Class-1
devices alone; Class-2 nodes simply participate as peers that happen to have
greater resources. Their absence does not prevent network formation, peer
discovery, or autonomous operation.

