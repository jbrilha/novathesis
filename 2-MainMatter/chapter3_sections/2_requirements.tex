%!TEX root = ../../template.tex

\section{System Requirements}
\label{sec:goals_requirements}

\subsection{Overview of \gls{ubabel}'s Approach}

% \gls{ubabel} addresses these concerns through a decentralized approach
% that aims to eliminate the dependency on centralized coordination, continuous
% connectivity and cloud infrastructure.
%
% Rather than treating infrastructure failures as an exceptional condition
% requiring failover mechanisms, our architecture focuses on autonomous
% \gls{P2P} connectivity as the baseline mode of operation, with
% infrastructure integration as an added benefit when available, rather than a
% hard dependency.
%
% This approach means rethinking multiple aspects of traditional \gls{iot}
% systems design: from multi-protocol communication strategies that adapt to the
% available mediums and device resources without centralized control, to
% gossip-based synchronization mechanisms that achieve coordination through local
% interactions, to lightweight data compression techniques to make the most out
% of the limited storage and battery available in each device.

In order to effectively overcome the difficulties exposed in the previous
chapter, we can highlight a set of \glspl{FR} that \gls{ubabel} must meet in
order to eliminate the \emph{dependency} on centralized coordination,
continuous connectivity, and cloud infrastructure; and a set of \glspl{NFR}
which would strengthen the overall quality of the framework and ensure it can
be widely adopted with ease.

\subsubsection*{\glslink{FR}{\glsentrylongpl{FR}}}
\label{subsubsec:func_reqs}

\paragraph{Infrastructure Independence:}

The system must maintain communication capabilities when traditional
infrastructure becomes unavailable. Rather than treating this
unavailability as an exceptional condition requiring failover mechanisms, our
architecture will focus on autonomous \gls{P2P} connectivity as the baseline
mode of operation, with infrastructure integration as an added benefit when
available, rather than a fundamental dependency.

\paragraph{Autonomous Operation:}

Devices will self-organize into functional networks, dispensing centralized
coordinator nodes. Instead, local \gls{P2P} gossip
interactions will be the basis for coordination and synchronization.

\paragraph{Multi-Radio Exploitation:}

Communication strategies will adapt to the available mediums and device
resources without centralized control. Communications will continue despite
disruption of individual radio interfaces through adaptive selection and
opportunistic switching. Interface diversity provides resilience
through redundancy, and this should be achieved with minimal impact on the
upper layers of the application (through abstractions).

\paragraph{Scalability:}

The system must support deployments ranging from small, localized networks to
large-scale, highly distributed collections. Rather than requiring global
knowledge of all participants, nodes organize into local groups where elected
leaders maintain state only for their immediate members. This area-based
approach -- analogous to \gls{OSPF}'s autonomous systems~\cite{rfc1247} --
distributes coordination load and enables networks to scale while keeping
per-node overhead bounded by local group size.

\paragraph{Resource Efficiency:}

Embedded platforms inevitably present strict resource constraints (i.e., memory
storage, processing, battery), thus all protocols and algorithms must execute
within these limitations. In order to enable this, adaptive selection of the
employed radio technologies will take into account device capabilities
and available resources.

\subsubsection*{\glslink{NFR}{\glsentrylongpl{NFR}}}
\label{subsubsec:nonfunc_reqs}

\paragraph{Optional Cloud Integration:}

When infrastructure \emph{is} available, it should be taken advantage of. Thus,
\gls{ubabel} will exploit it for tasks that are beyond the capabilities of
individual nodes or even gateways, such as data aggregation, analytics, and
long-term storage. Cloud integration thus becomes an optional -- and welcome --
addition to the repertoire of features, but does not itself become a
requirement for dependable operation.

\paragraph{Interoperability:}

Following optional cloud integration, widely used \gls{iot} communication
protocols such as \gls{MQTT} and \gls{CoAP} can be leveraged to enable
integration with existing \gls{iot} platforms and services when conditions
allow.
