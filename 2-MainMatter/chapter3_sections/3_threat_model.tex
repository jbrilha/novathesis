%!TEX root = ../../template.tex


\section{System Assumptions and Threat Model}
\label{sec:threat_model}
\todo{passar isto para depois do system model?}
\gls{ubabel}'s approach to security will reflect the priorities of
disaster-scenario deployments, prioritizing resilience over defending against
all possible attacks. We establish the following assumptions and threat
boundaries:

\subsection{Operational Assumptions}

\paragraph{Honest Majority:} Most nodes behave correctly and in accordance with
established protocol specifications. We assume adversaries lack the
coordination capabilities or resources to simultaneously compromise a majority
of the network.

\paragraph{Controlled Provisioning:}\todo{worth including sequer?}
Cryptographic primitives are distributed during initial device installation --
before any disasters. This is consistent with planned infrastructure
deployments where sensors are provisioned as part of their setup.

\paragraph{Local Adversaries:}
Attackers can monitor or disrupt communication within local regions but cannot
coordinate simultaneous attacks across the entire network. This reflects the
physical constraints of \gls{RF} jamming and the distributed nature of disaster
scenarios.

\subsection{Threat Model}

A sophisticated adversary capable of large-scale infrastructure compromise
would target communication systems regardless of deployed security mechanisms,
so our threat model prioritizes resilience against infrastructure
\emph{failure} while providing reasonable security against opportunistic
attacks during disasters.

We consider adversaries with the following capabilities, and briefly present
our corresponding defenses:

\begin{itemize}
    \item \textbf{Passive observation:} Can monitor wireless communications
        within radio range. \gls{AES} encryption prevents plaintext exposure.
    \item \textbf{Active interference:} Can disrupt specific protocols or
        channels locally. Multi-protocol diversity enables fallback to
        available mediums.
    \item \textbf{Node compromise:} Can physically remove devices from the
        network (denial of service). Flash encryption prevents secret
        extraction from captured devices.
    \item \textbf{Message injection:} Can attempt to inject forged messages
        without valid credentials. \gls{MAC} verification detects and rejects
        unauthorized transmissions.
    \item \textbf{Replay attacks:} Capture and retransmit previously valid
        messages. Sequence numbers or timestamp-based checks can detect and
        reject stale messages.
\end{itemize}

\subsection{Explicitly Out of Scope}

\inlinetodo{meter algo aqui ou not really? byzantine attacks, etc?}
