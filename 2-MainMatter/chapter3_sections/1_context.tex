%!TEX root = ../../template.tex

\section{Context}
\label{sec:cha3_context}

\subsection{The Problem}

Current \glspl{ISE} assume continuous infrastructure availability, an
assumption that proves rather catastrophic during disasters, in hazardous
environments, or when privacy requirements demand local autonomy
(Chapter~\ref{cha:introduction}).

The widely-adopted, cloud-centric architectural model creates single points of
failure in myriad ways: devices depend on remote servers for
control logic, local access points for Internet connectivity, and designated
coordinators for synchronization.

When infrastructure fails, whether due to large-scale
outages~\cite{thousandeyes2025aws} or disaster-related network collapse,
coordinated operation becomes impossible despite the local availability of
physical computational resources.

\subsection{Technical Challenges}

Throughout Chapter~\ref{cha:related_work} we identified the core challenges
that prevent existing approaches from fully addressing the overarching problem
of reliable autonomous operation:

\paragraph{Centralized Coordination:} Radio selection requires master nodes or
\gls{SDN} controllers. Topology management depends on gateway coordinators.
Time synchronization relies on master beacons. Authentication requires
persistent servers or group leaders. When these central elements fail or become
unreachable, coordination collapses despite local devices remaining
operational.

\paragraph{Single-Radio Operation:} Gossip-based synchronization assumes
homogeneous networks where all nodes share identical communication
characteristics, and topology maintenance protocols assume either persistent
connections or uniform radio properties ill-suited for embedded platforms with
access to multiple wireless technologies, whose usage can significantly
improve the resilience of these solutions.

\paragraph{Infrastructure dependencies:} Blockchain-based security requires
connectivity to distributed ledgers. Edge computing assumes eventual cloud
access. Hybrid architectures distribute processing but remain dependent on
infrastructure for coordination. Systems designed with "infrastructure as
fallback" still fail when that fallback becomes unavailable indefinitely.

\paragraph{Resource constraints:} Sophisticated coordination mechanisms exceed
embedded platform capabilities. Security schemes require specialized hardware
(\glspl{PUF}) or computationally expensive operations (bilinear pairings). The
tension between autonomous operation demands and limited computational, memory,
and energy budgets remains unresolved in existing approaches.\\

\gls{ubabel} will address these interconnected challenges through a unified
framework designed specifically for resource-constrained multi-radio
operation during infrastructure failures. The following sections detail the
design goals, targeted platforms, and design decisions that enable autonomous
coordination without central dependencies.
