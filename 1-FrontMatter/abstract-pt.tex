%!TEX root = ../template.tex
%%%%%%%%%%%%%%%%%%%%%%%%%%%%%%%%%%%%%%%%%%%%%%%%%%%%%%%%%%%%%%%%%%%%
%% abstract-pt.tex
%% NOVA thesis document file
%%
%% Abstract in Portuguese
%%%%%%%%%%%%%%%%%%%%%%%%%%%%%%%%%%%%%%%%%%%%%%%%%%%%%%%%%%%%%%%%%%%%

\typeout{NT FILE abstract-pt.tex}%

\glsresetall

Atualmente, pequenos sensores e atuadores podem formar a base de soluções
tecnológicas sofisticadas para infraestruturas \gls{iot} e aplicações de
domótica que monitorizam e controlam espaços, ou até promovem segurança em caso
de desastres, fornecendo orientação reativa e atempada. Infelizmente,
as soluções comerciais existentes nos domínios de \gls{iot} e domótica dependem
de infraestruturas centralizadas e baseadas na \emph{cloud} para operar, o que
na melhor das hipóteses pode levar à indisponibilidade destas soluções quando a
conectividade é perdida, e na pior -- em particular no contexto de gestão de
espaços críticos para a segurança -- potencial perda humana, uma vez que as
infraestruturas podem facilmente tornar-se inacessíveis durante um desastre.

Uma maneira de abordar esta questão é tirar partido de interações diretas entre
dispositivos e pequenos servidores \emph{edge} locais, explorando as diversas
tecnologias sem fios (i.e., Wi-Fi, \gls{BLE}, \gls{LoRa}, \gls{ZigBee},
\gls{ESP-NOW}) que permitem que tais sistemas permaneçam disponíveis mesmo
quando infraestruturas centralizadas ou acesso à Internet não estão. É possível
utilizar várias destas tecnologias para tornar os sistemas \gls{iot} e de
domótica mais robustos, selecionando automaticamente protocolos com base na sua
disponibilidade por dispositivo, congestão do meio, entre outros fatores. No
entanto, tirar partido desta heterogeneidade é desafiante: por um lado, porque
estes dispositivos normalmente têm poder computacional limitado, e por outro,
porque lidar com este nível de complexidade é uma tarefa difícil para os
programadores de aplicações.

Para superar estes desafios, neste trabalho planeamos estudar e desenvolver
mecanismos para explorar automaticamente essa heterogeneidade -- essencialmente
uma variante especializada da framework Babel (originalmente desenvolvida para
auxiliar no desenvolvimento de protocolos e aplicações distribuídos) para
dispositivos pequenos e integrados, tentativamente denominada \gls{ubabel}.
Planeamos validar e avaliar a nossa solução através de dois casos de uso
relevantes: um focado na aquisição de dados numa área ampla (e.g., campus
universitário), e outro focado num sistema de resposta a desastres no contexto
de um edifício, combinando aquisição de dados, decisões locais, e atuação
através de alertas e orientação de emergência.

\keywords{
	\glslink{iot}{\glsentrylong{iot}} (\glslink{iot}{\glsentryshort{iot}}) \and
	Operação Autónoma \and
	Comunicação Sem Fios Híbrida \and
	Resiliência a Desastres \and
	Sistemas Embutidos \and
	Redes Mesh \gls{P2P}
}
