%!TEX root = ../template.tex
%%%%%%%%%%%%%%%%%%%%%%%%%%%%%%%%%%%%%%%%%%%%%%%%%%%%%%%%%%%%%%%%%%%%
%% abstract-pt.tex
%% NOVA thesis document file
%%
%% Abstract in Portuguese
%%%%%%%%%%%%%%%%%%%%%%%%%%%%%%%%%%%%%%%%%%%%%%%%%%%%%%%%%%%%%%%%%%%%

\typeout{NT FILE abstract-pt.tex}%

\glsresetall

Sistemas atuais de \gls{iot} e domótica dependem fundamentalmente de
disponibilidade contínua de infraestrutura, o que cria vulnerabilidades
críticas quando conectividade à \emph{cloud} falha, pontos de acesso tornam-se
inalcançáveis, ou coordenadores centralizados deixam de responder.

Resolver esta dependência é particularmente desafiante quando o alvo são
plataformas com recursos limitados, devido à natureza conflituosa de operação
autónoma e as capacidades computacionais, de memória, e de energia disponíveis.
A heterogeneidade de protocolos de comunicação disponíveis (i.e., Wi-Fi,
\gls{BLE}, \gls{LoRa}, \gls{ZigBee}, \gls{ESP-NOW}) introduz complexidade
adicional mas -- se utilizada corretamente -- pode ajudar a mitigar a
dependência em infraestrutura ao providenciar resiliência e autonomia através
de variedade.

Neste trabalho propômos \gls{ubabel}, uma \emph{framework} de comunicação
\emph{multi-protocolo} destinada a abordar as dificuldades de coordenação
descentralizada em dispositivos embutidos comuns (família ESP32, Raspberry Pi
Pico). Explorando os meios de comunicação disponíveis, planeamos possibilitar
operação resiliente, descentralizada, e independente de infraestruturas com:
seleção adaptativa de protocolos baseada em condições operacionais, requisitos
de \emph{\gls{QoS}}, e de capacidades dos dispositivos; descoberta autónoma de
\emph{peers}; sincronização temporal baseada em \emph{gossip}; e técnicas
eficientes de redução de dados.

O sistema proposto é desenhado para operação completamente autónoma,
mas permitirá integração natural\todo{maybe not "natural" but seamless é kinda estranho traduzir} com o ecosistema
Babel~\cite{fouto2022babel}, possibilitando a implementação e formação de
sistemas heterógeneos onde dispositivos com recursos limitados na \emph{edge}
interoperam com nós computacionais \emph{opcionais} para capacidades
adicionais quando disponíveis.

\keywords{
	\glslink{iot}{\glsentrylong{iot}} (\glslink{iot}{\glsentryshort{iot}}) \and
	Operação Autónoma \and
	Comunicação Multi-Protocolo \and
	Resiliência a Desastres \and
	Sistemas Embutidos \and
	Redes Mesh \gls{P2P} \and
}
