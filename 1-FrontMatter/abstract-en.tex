%!TEX root = ../template.tex
%%%%%%%%%%%%%%%%%%%%%%%%%%%%%%%%%%%%%%%%%%%%%%%%%%%%%%%%%%%%%%%%%%%%
%% abstract-en.tex
%% NOVA thesis document file
%%
%% Abstract in English
%%%%%%%%%%%%%%%%%%%%%%%%%%%%%%%%%%%%%%%%%%%%%%%%%%%%%%%%%%%%%%%%%%%%

\typeout{NT FILE abstract-en.tex}%

Nowadays, small sensors and actuators can form the basis for sophisticated
technological solutions for \gls{iot} infrastructures and domotics applications
that monitor and control spaces, or even promote safety in case of disasters,
by providing guidance in a reactive and timely manner. Unfortunately, existing
commercial solutions in the domains of \gls{iot} and domotics are dependent on
centralized and cloud-based infrastructures to operate, which at best can lead
to the unavailability of these solutions when connectivity is lost, and at
worst -- particularly in the context of safety-critical space management --
potential human loss, as infrastructures can easily become inaccessible during
a disaster.

One way to address this is to take advantage of direct interactions between
devices and small on-premises edge servers, exploiting the myriad wireless
technologies (i.e., Wi-Fi, \gls{BLE}, \gls{LoRa}, \gls{ZigBee}, \gls{ESP-NOW})
that allow such systems to remain available even when centralized
infrastructures or Internet access are not. One can leverage multiple of these
technologies to make \gls{iot} and domotics systems more robust, by
automatically selecting protocols based on their availability per device,
medium congestion, among other factors. However, taking advantage of this
heterogeneity is challenging: on one hand, because these devices usually have
limited computational power, and on the other, because tackling this level of
complexity is a hard task for application developers.

To overcome these challenges, in this work we plan to study and develop
mechanisms to automatically exploit that heterogeneity -- essentially a
specialized variant of the Babel framework (originally developed to assist in
the development of distributed protocols and applications) for small and
integrated devices, tentatively named \gls{ubabel}. We plan to validate and
evaluate our solution through two relevant use cases: one focused on data
acquisition within a large area (e.g., university campus), and another focused
on a disaster-response system in the context of a building, combining data
acquisition, local decisions, and actuation through alerts and emergency
guidance.

\keywords{
	\glslink{iot}{\glsentrylong{iot}} (\glslink{iot}{\glsentryshort{iot}}) \and
	Autonomous Operation \and
	Multi-Radio Communication \and
	Disaster Resilience \and
	Embedded Systems \and
	\gls{P2P} Mesh Networks
}
